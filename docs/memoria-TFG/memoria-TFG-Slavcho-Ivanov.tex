\documentclass[a4paper,11pt]{book}
%\documentclass[a4paper,twoside,11pt,titlepage]{book}
\usepackage{array}
\usepackage{appendix}
\usepackage{cite}
\usepackage{dirtree}
\usepackage{enumerate} % enumerados
\usepackage{multirow}
\usepackage{listings}
\usepackage{longtable}
\usepackage{float}
\usepackage{subfig}
\usepackage{url}
\usepackage[utf8]{inputenc}
\usepackage[spanish]{babel}
\usepackage[sort&compress]{natbib}

\decimalpoint
\usepackage{dcolumn}
\newcolumntype{.}{D{.}{\esperiod}{-1}}
\makeatletter
\addto\shorthandsspanish{\let\esperiod\es@period@code}
\makeatother


%\usepackage[chapter]{algorithm}
\RequirePackage{verbatim}
%\RequirePackage[Glenn]{fncychap}
\usepackage{fancyhdr}
\usepackage{graphicx}
\usepackage{afterpage}

\usepackage{longtable}

\usepackage[pdfborder={000}]{hyperref} %referencia

% ********************************************************************
% Re-usable information
% ********************************************************************
\newcommand{\myTitle}{Desarrollo de un simulador de escenarios con usuarios móviles para evaluar sistemas de recomendación}
\newcommand{\myDegree}{GRADO EN INGENIERÍA INFORMÁTICA}
\newcommand{\TFG}{TRABAJO FIN DE GRADO}
\newcommand{\myName}{Slavcho Georgiev Ivanov}
\newcommand{\myProf}{Sergio Ilarri Artigas}
\newcommand{\myFaculty}{Escuela de Ingeniería y Arquitectura}
\newcommand{\myFacultyShort}{EINA}
\newcommand{\myArea}{Área de Lenguajes y Sistemas Informáticos}
\newcommand{\myDepartment}{Departamento de Informática e Ingeniería de Sistemas}
\newcommand{\myUni}{\protect{Universidad de Zaragoza}}
\newcommand{\myLocation}{Zaragoza}
\newcommand{\myTime}{\today}
\newcommand{\myVersion}{Version 0.1}


\hypersetup{
pdfauthor = {\myName (email (en) ugr (punto) es)},
pdftitle = {\myTitle},
pdfsubject = {},
pdfkeywords = {palabra_clave1, palabra_clave2, palabra_clave3, ...},
pdfcreator = {LaTeX con el paquete ....},
pdfproducer = {pdflatex}
}

\usepackage{url}
\usepackage{colortbl,longtable}
\usepackage[stable]{footmisc}

\pagestyle{fancy}
\fancyhf{}
\fancyhead[LO]{\leftmark}
\fancyhead[RE]{\rightmark}
\fancyhead[RO,LE]{\textbf{\thepage}}
\renewcommand{\chaptermark}[1]{\markboth{\textbf{#1}}{}}
\renewcommand{\sectionmark}[1]{\markright{\textbf{\thesection. #1}}}

\setlength{\headheight}{1.5\headheight}

\newcommand{\HRule}{\rule{\linewidth}{0.5mm}}
%Definimos los tipos teorema, ejemplo y definición podremos usar estos tipos
%simplemente poniendo \begin{teorema} \end{teorema} ...
\newtheorem{teorema}{Teorema}[chapter]
\newtheorem{ejemplo}{Ejemplo}[chapter]
\newtheorem{definicion}{Definición}[chapter]

\definecolor{gray97}{gray}{.97}
\definecolor{gray75}{gray}{.75}
\definecolor{gray45}{gray}{.45}
\definecolor{gray30}{gray}{.94}

\lstset{ frame=Ltb,
     framerule=0.5pt,
     aboveskip=0.5cm,
     framextopmargin=3pt,
     framexbottommargin=3pt,
     framexleftmargin=0.1cm,
     framesep=0pt,
     rulesep=.4pt,
     backgroundcolor=\color{gray97},
     rulesepcolor=\color{black},
     %
     stringstyle=\ttfamily,
     showstringspaces = false,
     basicstyle=\scriptsize\ttfamily,
     commentstyle=\color{gray45},
     keywordstyle=\bfseries,
     %
     numbers=left,
     numbersep=6pt,
     numberstyle=\tiny,
     numberfirstline = false,
     breaklines=true,
   }
 
% minimizar fragmentado de listados
\lstnewenvironment{listing}[1][]
   {\lstset{#1}\pagebreak[0]}{\pagebreak[0]}

\lstdefinestyle{CodigoC}
   {
	basicstyle=\scriptsize,
	frame=single,
	language=C,
	numbers=left
   }
\lstdefinestyle{CodigoC++}
   {
	basicstyle=\small,
	frame=single,
	backgroundcolor=\color{gray30},
	language=C++,
	numbers=left
   }

 
\lstdefinestyle{Consola}
   {basicstyle=\scriptsize\bf\ttfamily,
    backgroundcolor=\color{gray30},
    frame=single,
    numbers=none
   }


\newcommand{\bigrule}{\titlerule[0.5mm]}


%Para conseguir que en las páginas en blanco no ponga cabecerass
\makeatletter
\def\clearpage{%
  \ifvmode
    \ifnum \@dbltopnum =\m@ne
      \ifdim \pagetotal <\topskip
        \hbox{}
      \fi
    \fi
  \fi
  \newpage
  \thispagestyle{empty}
  \write\m@ne{}
  \vbox{}
  \penalty -\@Mi
}
\makeatother

\usepackage{pdfpages}
\begin{document}
\begin{titlepage}
 
 
\newlength{\centeroffset}
\setlength{\centeroffset}{-0.5\oddsidemargin}
\addtolength{\centeroffset}{0.5\evensidemargin}
\thispagestyle{empty}

\noindent\hspace*{\centeroffset}\begin{minipage}{\textwidth}

\centering
\includegraphics[width=0.9\textwidth]{imagenes/diislogoblanco_0.png}\\[1.4cm]

\textsc{ \Large TRABAJO FIN DE GRADO\\[0.2cm]}
\textsc{ GRADO EN INGENIERÍA INFORMÁTICA}\\[1cm]
% Upper part of the page
% 
% Title
{\Huge\bfseries Simulador de recomendaciones\\
}
\noindent\rule[-1ex]{\textwidth}{3pt}\\[3.5ex]
%{\large\bfseries Subtitulo del Proyecto}
\end{minipage}

\vspace{2.5cm}
\noindent\hspace*{\centeroffset}\begin{minipage}{\textwidth}
\centering

\textbf{Autor}\\ {Slavcho Georgiev Ivanov}\\[2.5ex]
\textbf{Director}\\
{Sergio Ilarri Artigas}\\[2cm]
\includegraphics[width=0.3\textwidth]{imagenes/logo-eina.png}\\[0.1cm]
\textsc{Escuela de Ingeniería y Arquitectura}\\
\textsc{Área de Lenguajes y Sistemas Informáticos}\\
\textsc{Departamento de Informática e Ingeniería de Sistemas}\\
\textsc{---}\\
Zaragoza, Julio de 2016
\end{minipage}
%\addtolength{\textwidth}{\centeroffset}
%\vspace{\stretch{2}}
\end{titlepage}



\chapter*{}

\cleardoublepage
\thispagestyle{empty}

\begin{center}
{\large\bfseries \myTitle}\\
\end{center}

\vspace{0.7cm}
\noindent{\textbf{Resumen}}\\

Los sistemas de recomendación proporcionan sugerencias acerca de elementos que pueden resultar de interés para el usuario (hoteles, restaurantes, libros, películas, etc.). En este TFG se pretende desarrollar un simulador de escenarios con usuarios móviles (mapas de ciudades con objetos móviles y estáticos) que permita la evaluación de algoritmos de recomendación. 

\cleardoublepage


\thispagestyle{empty}


\chapter*{Agradecimientos}
\thispagestyle{empty}

       \vspace{1cm}


Me gustaría agradecer este Trabajo Fin de Grado a todas las personas que lo han echo posible con su apoyo y dedicación.

\vspace{1cm}
En su primer lugar a mi director Sergio Ilarri por la oportunidad que me ha dado para realizar este proyecto, su paciencia y ayuda, sin la cual este proyecto no hubiera sido posible. A mis compañeros y amigos de clase, con los que he compartido estos años de carrera, por hacer que los momentos de estudio y prácticas fuesen agradables y amenos. A mi familia y amigos más cercanos, por su paciencia y por motivarme para seguir adelante en los momentos más complicados.

\vspace{1cm}
Y por supuesto a la Universidad de Zaragoza y a todos aquellos profesores de lo que he aprendido tanto a los largo de estos años.

\cleardoublepage

\tableofcontents % indice de contenidos

\cleardoublepage
\addcontentsline{toc}{chapter}{Lista de figuras} % para que aparezca en el indice de contenidos
\listoffigures % indice de figuras

\cleardoublepage
\addcontentsline{toc}{chapter}{Lista de tablas} % para que aparezca en el indice de contenidos
\listoftables % indice de tablas

\frontmatter
\tableofcontents
\listoffigures
\listoftables

%
\mainmatter
\setlength{\parskip}{5pt}

\chapter{Introducción}
\thispagestyle{empty}

En este capítulo se mostrará la motivación existente para la realización de este Trabajo Fin de Grado, los objetivos que han sido marcados por el proyecto, las librerías y herramientas utilizadas para su elaboración, el modelo de trabajo seleccionado y también se analizará el trabajo relacionado. Finalmente se mostrará la estructura seguida en este documento.

\section{Motivación del proyecto}
\thispagestyle{empty}

       \vspace{1cm}

Han sido varias las razones que me llevaron a elegir desarrollar este Trabajo Fin de Grado. La primera y principal ha sido el interés personal en los sistemas de recomendaciones y su amplia aplicación en sistemas comerciales. Por otro lado, realizar un proyecto complejo, partiendo desde cero y sin tener ningún conocimiento particular de este ámbito, suponía un gran reto que deseaba afrontar porque me permitiría ampliar mis conocimientos en campos diversos como Ingeniería del Software, Arquitecturas de Software, Inteligecia Artificial etc., de las que poseía unos conocimientos limitados. Además, consideré que la experiencia y conocimientos que adquiriría en este proyecto aumentarian mis posibilidades de desarrollar mi carrera profesional en este ámbito.

\section{Objetivos}
\thispagestyle{empty}

       \vspace{1cm}

El Trabajo Fin de Grado que se describe en este documento tiene los siguientes objetivos:

\begin{itemize}
	\item objetivo 1
	\item objetivo 2
\end{itemize}

%Además de estos objetivos marcados por la propuesta del Trabajo Fin de Grado, también se ha tenido en cuenta como objetivo lograr el simulador ...


\section{Herramientas utilizadas}
\thispagestyle{empty}

       \vspace{1cm}

En esta sección se listan las tecnologias, libererías externas y herramientas utilizadas para el desarrollo del proyecto acompañada de una breve descripción del porqué de su uso.

\section{Modelo de proceso seleccionado}
\thispagestyle{empty}

       \vspace{1cm}

El modelo de trabajo seleccionado está basado en el modelo de espiral. Las actividades de este modelo forman una espiral de tal forma que cada iteración representa un conjunto de actividades. Se ha elegido este modelo de trabajo porque nos permitiría integrar el desarrollo con el mantenimiento y evaluar en cada iteración si dichos requisitos siguen encajando de lo que se esperaba de la aplicación para conseguir los objetivos propuestos. De esta forma se reduce el riesgo del proyecto y se incorporan objetivos de calidad.


\section{Trabajos relacionados}
\thispagestyle{empty}

       \vspace{1cm}

Poner los trabajos relacionados


\section{Estructura de la memoria}
\thispagestyle{empty}
       \vspace{1cm}

El contenido de la memoría está distribuido de la siguiente forma:

\begin{itemize}
	\item En el capítulo 2 se expone el trabajo desarrollado para la elaboración del simulador
	\item En el capítulo 3 se analiza la posible explotación del simulador como método para probar diferentes tipos de algoritmos de recomendaciones, También se muestra el rendimiento obtenido del simulador
	\item En l capítulo 4 se muestran las conclusione del proyecto y el posible trabajo futuro de cada a mejorar el simulador 
\end{itemize}
%
\chapter{Trabajo desarrollado}
\thispagestyle{empty}

En este capítulo se explican las funcionalidades básicas del simulador desarrollado centradose únicamente en los aspectos más importantes. Para profundizar más sobre estos aspectos debe acudir a los anexos.

\section{Resumen del simulador}
\thispagestyle{empty}

Se trata de un sistema de colaboración abierta distribuida que permite configurar distintos escenarios con objetos móviles y estáticos sobre mapas de ciudades reales obtenidos a partir del servicio de mapas de OpenStreetMap. 

El simulador de escenarios está basado en el simulador Mavsim desarrollado por el Grupo de Sistemas de Información Distribuidos de la Universidad de Zaragoza utilizado para la simulación de VANETs en el cual hay muchos vehículos distribuidos en una amplia zona geografica.

\begin{figure}[H]
\centering\includegraphics[scale=0.3]{imagenes/resumen-simulador.jpg}
\caption{Simulación en Actur, Zaragoza con un solo usuario}
\label{c2_trama}
\end{figure}

Puede ser usado a través de cualquier dispositovo (PC, tablet, móvil etc.) con conexión a Internet y un navegador web. Permite a los usuarios crear sus propios mapas y escenarios. Durante la creación de una escena el usuario elige cual es la cuidad donde se realiza la simulación, el recomendador a utilizar, si el mapa es colaborativo o no, introducir los objetos estáticos y configurar cuales son los objetos móviles y sus rutas. 

Para realizar una simulación el usuario tiene que buscar y seleccionar el mapa y escenario donde moverse tanto para obtener recomendaciones como para realizar votaciones sobre los distintos objetos de este entorno.

\section{Análisis de requisitos}
\thispagestyle{empty}

\subsection{Requisitos funcionales}
\thispagestyle{empty}

\begin{table}[htbp]
\begin{center}
\begin{tabular}{|l|l|}
\hline
Código de requisito & Descripción \\
\hline \hline
RF-1  & Descripcion \\ \hline
RF-1  & Descripcion \\ \hline
RF-1  & Descripcion \\ \hline
RF-1  & Descripcion \\ \hline
RF-1  & Descripcion \\ \hline
RF-1  & Descripcion \\ \hline
RF-1  & Descripcion \\ \hline
\end{tabular}
\caption{Requisitos funcionales}
\label{tabla:requisitosFuncionales}
\end{center}
\end{table}


\subsection{Requisitos no funcionales}
\thispagestyle{empty}

\begin{table}[htbp]
\begin{center}
\begin{tabular}{|l|l|}
\hline
Código de requisito & Descripción \\
\hline \hline
RNF-1  & Descripcion \\ \hline
RNF-1  & Descripcion \\ \hline
RNF-1  & Descripcion \\ \hline
RNF-1  & Descripcion \\ \hline
RNF-1  & Descripcion \\ \hline
RNF-1  & Descripcion \\ \hline
RNF-1  & Descripcion \\ \hline
\end{tabular}
\caption{Requisitos no funcionales}
\label{tabla:requisitosNoFuncionales}
\end{center}
\end{table}

\section{Arquictura del sistema}
\thispagestyle{empty}

La arquitectura del sistema consta de cliente o navegador web, servidor web Node.js, servidor de recomendaciones y base de datos mongoDB (figura \ref{arquitecturaComponentes}).

El la figura \ref{arquitecturaComponentes} observamos que el navegador web se conecta al servidor Node.js mediante dos maneras: la primera es HTTP y la segunda es un sistema bidireccional dirigido por eventos. Las funcionalidades como creación de escenas, busqueda de mapas etc. están desarrollados sobre una REST API y el intercambio de mensajes JSON.

El sistema bidireccional dirigido por eventos es utilizado por una parte durante la simulación, para reflejar los eventos generados por un usuario al resto de usuarios, y por otra para integrar el navegador, el servidor Node.js y el recomendador. De esta manera conseguimos compartir informácion entre los distintos componentes sin que estos los hayan solicitado evitando muchas peticiones innecesarias.

\begin{figure}[H]
\centering\includegraphics[scale=0.4]{imagenes/arquitectura-componentes.png}
\caption{Arquitectura de componentes del sistema}
\label{arquitecturaComponentes}
\end{figure}

\subsection{Arquitectura del front end}
\thispagestyle{empty}

Para el desarrollo del front end se han utilizado los frameworks Angular.js\footnote{framework que nos permite desarrollar aplicaciones de una sola página} y bootstrap\footnote{framework que nos ofrece un sistema de componentes reutilizables y adaptables a la pantalla del dispositivo}. Se ha decidido utilizar estas tecnologias porque nos ofrece varias ventajas: ahorro de recursos\footnote{angular.js va transmitiendo las vistas de la interfaz gráfica y las cachea al lado del cliente para ser reutilizadas posteriormete. Vuelve a solicitar una vista si y solo si esta ha sufrido algún cambio en el servidor}, mejora de la productividad y la posibilidad de realizar una simulación sobre dispositivos móviles\footnote{esto nos da la oportunidad de utilizar la posición geográfica del usuario para obtener recomendaciones en el entorno de una cuidad real. De está manera obtenemos datos reales y muchas más precisión a la hora de evaluar los algoritmos de recomendaciones} con el mismo código fuente.

La arquitectura del front end está basada en el patrón Modelo-Vista-Controllador de tal forma que para cada vista existe un controlador que contiene la lógica de negocio de esta. El controlador tambien es el encargado de establecer comunicación con el back end. Esta comunicación se realiza mediante los llamados Servicios\footnote{es pequeña fabrica de funciones y objetos inyectada en los controladores} de Angular.js y la REST API del back end. Los Servicios de Angular.js son muy necesarios y útiles ya que nos permiten crear un envoltorio sobre la REST API que nos ofrece el back end y de esta forma centralizar las llamas a la API.

\begin{figure}[H]
\centering\includegraphics[scale=0.6]{imagenes/arquitectura-front-end.png}
\caption{Arquitectura del front end}
\label{arquitecturaFrontEnd}
\end{figure}

\subsection{Arquitectura del back end}
\thispagestyle{empty}

Para el desarrollo del back end se han utilizado Node.js\footnote{javascript al lado del servidor}, Express\footnote{modulo de Node.js que nos ofrece la posiblidad de desarrollar una REST API}, socket.io\footnote{sistema bidireccional dirigido por eventos} y mongoose \footnote{modelado de objetos sobre mongoDB}. Se ha decidido utilizar estas tecnología por que nos ofrecen las siguientes ventajas: mejora la productividad a la hora de desarrollar el back end, nos permite desarrollar un back end ligero que consuma pocos recursos y el modulo sockets.io nos ofrece la posibilidad de desarrollar un sistema bidireccional dirigido por eventos.

La arquitectura del back end se basa en la filosofía de desarrollo de aplicaciones con Node.js y Express y consiste de la siguiente estructura de directorios:

\dirtree{%
.1 Simulator.
.2 app.js.
.2 package.json.
.2 bin.
.3 www.
.2 models.
.2 public.
.3 images.
.3 javascript.
.4 angular.
.4 bootstrap.
.4 jquery.
.4 openlayers.
.4 socketIO.
.3 stylesheets.
.3 views.
.4 configurations.
.4 maps.
.4 settings.
.4 index.html.
.4 register.html.
.3 index.html.
.2 routes.
.3 configuratios.js.
.3 maps.js.
.3 simulation.js.
.3 user.js.
}

\newpage

A continuación vamos a ver más detalladamente cual es la funcion de cada elemento de este directorio:
\begin{itemize}
	\item app.js: centraliza las configuraciones de nuestra aplicación como por ejemplo en que puerto arranca el servidor, establecer conexiones con la base de datos, configuraciones del router\footnote{el router de Node.js es el encargado del direccionamiento de las peticiones y hace referencia a la definición de puntos finales de aplicación (URI) y cómo responden a las solicitudes de cliente} etc.
	\item package.json: es un gestor de paquetes y contiene los modulos que se están utilizando en nuestra aplicación.
	\item public: es un directorio que contiene la parte visual, es decir el front end. Podemos ver que este contiene varios subdirectorios:
	\begin{itemize}
	\item en images se ubican las imagenes o iconos usados en la aplicación.
	\item en javascript se ubican los frameworks Javascript usados en en el front end como Angular.js, Bootstrap, OpenLayers etc. En el subdirectorio angular podemos encontrar las configuraciones de Angular.js, los controladores de las vistas, los Servicios etc.
	\item en views se ubican las vistas del front end.
	\item en stylesheets están ubicadas las hojas de estilos
	\end{itemize}
	\item en routes se ubican las distinas rutas de Express. En nuestro caso tenemos una para cada menú de la aplicación. Por ejemplo todas las operaciones referentes al menú Maps se encuentra en el fichero maps.js etc.
\end{itemize}

\subsection{Arquitectura de recomendador}
\thispagestyle{empty}

El desarrollo del recomendador está realizado con Java 7, la librería Apache Mahout y socket.io-client. El recomendador desarrollado es un recomendador pull\footnote{tipo de recomendador en el cual los usuarios solicitan recomendaciones} de ejemplo basado en los usuarios (User based recommender). 

Aunque sea un recomendador de ejemplo este está pensado para se expandido, tanto para otro tipo de recomendadores (como pueden ser los recomendadores de tipo push\footnote{tipo de recomendador que realiza recomendaciones sin que el usuario los haya solicitado}) como para la implementación de nuevos tipos de estrategias para el recomendador de tipo pull.

Durante el arranque el servidor del recomendador lanza un hilo por cada tipo de recomendador. En nuestro caso solo lanza un hilo que se corresponde al servidor de tipo pull. Este hilo es el que contiene los eventos que invoca el simulador de escenarios. En la figura \ref{diagramaEventos} observamos que tenemos solo dos eventos: uno para recuperar los tipos de implementaciones\footnote{en nuestro caso es User based recommender pero existen otros tipos como Item based recommender} y otro para realizar las recomendaciones.

\begin{figure}[H]
\centering\includegraphics[scale=0.6]{imagenes/diagrama-de-eventos.png}
\caption{Diagrama de eventos}
\label{diagramaEventos}
\end{figure}

El evento recommend es el que se dispara cuando uno de los usuarios solicita una recomendación. Para que se puedan utilizar distintos tipos de implementaciones del recomendador de tipo pull se ha implementado un patrón de diseño de tipo Strategy. Este patrón de diseños nos permite cambiar de estrategia de recomendación en tiempo de ejecución:

\begin{figure}[H]
\includegraphics[scale=0.7]{imagenes/uml.png}
\caption{Diagrama uml del patrón de diseño de tipo Strategy}
\label{diagramaUMLStrategy}
\end{figure}

\section{Menús del simulador de escenarios}
\thispagestyle{empty}

Como en todas las aplicaciones, el simulador de escenarios cuenta con un sistema de menús que dan acceso a las distintas opciones del simulador. En este caso se ha seguido la paradigma WIMP para la organización de los menús y distinas opciones (figura \ref{mapaNavegacion}). Existen 4 tipos de caminos organizados en forma de árbol:

\begin{itemize}
	\item busqueda y gestión (creación, edición y borrados) de mapas y escenas
	\item simulación de un mapa y escena
	\item gestión de tipos de recomendadores, objetos estáticos\footnote{objetos que no cambian de posición a medida que pasa el tiempo} y dinámicos\footnote{objetos que cambian de posición a medida que pasa el tiempo. Tienen una ruta difinida durante la creación de la escena}
	\item configuraciones del perfil del usuario
\end{itemize} 

\begin{figure}[H]
\centering\includegraphics[scale=0.7]{imagenes/mapa-navegacion.png}
\caption{Mapa de navegación}
\label{mapaNavegacion}
\end{figure}

\newpage

\section{Navegación por estima}
\thispagestyle{empty}

La navegación por estima es una técnica que se aplica en el cliente. Consiste en procesar en cada ciclo el estado de los objetos móviles. Se trata de una técnica análitica utilizada en la náutica para la navegación y situación de los barcos y se tienen en cuenta los siguientes elementos: la situación actual, rumbo y velocidad. Es decir, sabiendo la velocidad, el rumbo de la nave y el tiempo transcurido se puede estimar la posición de la misma al cabo del tiempo. 

Con este método conseguimos calcular cual es la siguiente posición geográfica donde tenemos que colocar un objeto móvil al acabo de un tiempo (el tiempo de refresco de la pántalla). Como ventaja conseguimos disminuir el error en el cálculo de las posiciones de los objetos móviles. Así obtenemos movimientos muy precisos incluso en grafos de movimientos con nodos muy cercanos. Para evitar problemas en estos casos se han puesto tiempos bajos de refresco de la pántalla.

\section{Diseño final}
\thispagestyle{empty}

%
\chapter{Explotación}

En este capítulo se expone la explotación del simulador de escenarios como método para la evaluación de sistemas de recomendaciones, el trabajo realizado para permitirlo y el rendimiento obtenido del simulador. Para más información consulte el manual del usuarios disponible en los anexos.

\section{Motivación}

Debido a la gran cantidad de información resulta difícil para los usuarios elegir entre todas las alternativas existentes en los resultados mostrados por un sistema. Por ello, para facilitar la elección se utilizan los llamados sistemas de recomendación. Estos sistemas resultan de gran interés para las empresas desde punto de vista económico, ya que hacen llegar a sus cliente recomendaciones de productos relevantes para ellos. Ademas, también resultan de interés para los usuarios ya que actúan como filtro y les ofrecen información personalizada que les resulta de interés.

No obstante, el diseño de sistemas de recomendación se enfrenta a problema como el arranque en frió (cold
start problem) o el problema de opiniones artificiales o manipuladas (spam). Esto conlleva que el sistema de recomendación muestra información no relevante para el usuario y este dejará de confiar en él y abandonarlo. 

Por esto surge la necesidad de que los sistemas de recomendaciones puedan evaluarse y calibrarse correctamente. Para que esto pueda llevarse a cabo necesitamos recolectar datos reales de escenarios reales para su posterior análisis. Este análisis consiste en calcular el error cometido por el sistema de recomendación. Una forma de cuantificar el error es con la MAE (Medium Absolute Error) que consiste en restar el valor asignado por el usuario del valor estimado por el sistema de recomendación. 

\section{Definir un escenario realista}

En este apartado se expone como definir un escenario realista con datos de restaurantes reales de la ciudad San Luis Potosí de México.

\subsection{Paso 1: obtener datos de restaurantes reales}

Lo primero que vamos a hacer es conseguir datos reales de restaurantes reales. Por esto accedemos al repositorio de Center for Machine Learning and Intelligent Systems y descargamos el siguiente archivo comprimido: \href{https://archive.ics.uci.edu/ml/machine-learning-databases/00232/RCdata.zip}{https://archive.ics.uci.edu/ml/machine-learning-databases/00232/RCdata.zip}

Una vez que hayamos descargado y descomprimido el fichero vemos que este contiene varios ficheros csv. A nosotros nos interesan los ficheros geoplaces2.csv y rating\_final.csv. El primero contiene los datos de los restaurantes y es el que vamos a importar al crea un escenario y el segundo contiene datos con valoraciones de diferentes usuarios y el que vamos a usar como datos de entrenamiento para el recomendador de ejemplo desarrollado.

\subsection{Paso 2: crear un mapa nuevo o editar un mapa existente}

Antes de todos tenemos que crear un mapa nuevo o editar uno existente para poder asociarle el escenario que vamos a crear a continuación. Por esto tenemos que usar alguna de las dos opciones:
\begin{itemize}
	\item crear un mapa nuevo: opción  Maps $\rightarrow$ new map. Para más información por favor consulte el anexo \ref{sec:crearMapa}.
	\item editar un mapa existente: opción Maps $\rightarrow$ maps search y en la lista de resultados pulsamos sobre el icono de editar imagen. Para más información por favor consulte el anexo \ref{sec:editarMapasEscenas}.
\end{itemize} 

En este caso se ha optado por crear un nuevo mapa con el nombre México ya que los datos que hemos descargado son de la ciudad San Luis Potosí de México. En la figura \ref{mapaMexico} podemos ver una captura del resultado de creación del mapa:

\begin{figure}[H]
	\centering\includegraphics[scale=0.36]{imagenes/crear-mapa-nuevo-mexico.jpg}
	\caption{Creación de un mapa nuevo en México con datos reales}
	\label{mapaMexico}
\end{figure}

\subsection{Paso 3: crear un escenario realista}

La creación de escenarios funciona como un asistente de configuración. A continuación se muestra los pasos que hay que realizar para configurar un escenario realista:

\subsubsection{Paso 1: definir el nombre del escenario y elegir el recomendador}

En el primer paso de la creación de un escenario consiste en poner un nombre del escenario y elegir que recomendador vamos a usar en dicho escenario. En la figura \ref{paso1DefinirNombreYrecomendador} podemos una captura de este paso:

\begin{figure}[H]
	\centering\includegraphics[scale=0.45]{imagenes/explotacion/1.jpg}
	\caption{Paso 1: definir un nombre y recomendador para el escenario realista}
	\label{paso1DefinirNombreYrecomendador}
\end{figure}

\subsubsection{Paso 2: definir los limites del escenario}

En segundo paso de la creación del escenario realista consiste en definir los limites geográficos del escenario. Por esto lo primero que tenemos que hacer es usar el buscador para buscar la ciudad donde se va a realizar la simulación. Una vez que hayamos encontrado y seleccionado la cuidad definiremos la esquina superior izquierda ya la esquina inferior derecha. En la figura \ref{paso2definirLimitesEscenarioRealista} vemos el resultado:

\begin{figure}[H]
	\centering\includegraphics[scale=0.45]{imagenes/explotacion/2.jpg}
	\caption{Paso 2: definir los limites del escenario realista}
	\label{paso2definirLimitesEscenarioRealista}
\end{figure}

\subsubsection{Paso 3: importar los datos de los restaurantes}

El siguiente paso consiste definir los restaurantes. El simulador dispone de varias opciones mediante las cuales es posible hacer esto. Nosotros vamos a importar los restaurantes del fichero geoplaces2.csv que hemos descargado previamente. Antes de cargar los datos el contenido del fichero se pre visualiza en la pantalla (figura \ref{paso3definirLosRestaurantes})

\begin{figure}[H]
	\centering\includegraphics[scale=0.35]{imagenes/explotacion/3.jpg}
	\caption{Paso 3: definir los restaurantes sobre el escenario}
	\label{paso3definirLosRestaurantes}
\end{figure}

\newpage

Una vez importados los datos del fichero vemos que se muestra una lista con los restaurantes importados. Podemos ver esta lista con la figura \ref{paso31definirLosRestaurantes}:

\begin{figure}[H]
	\centering\includegraphics[scale=0.4]{imagenes/explotacion/4.jpg}
	\caption{Paso 3: lista de restaurantes definidos sobre el escenario}
	\label{paso31definirLosRestaurantes}
\end{figure}

\subsubsection{Paso 4: definir los objetos móviles}

El ultimo paso consiste en definir los objetos móviles. El simulador dispone de varias opciones mediante cuales se pueden definir objetos móviles. En este caso se han generado automáticamente (en la figura \ref{paso4ListaObjetosMoviles} podemos ver el resultado).

\begin{figure}[H]
	\centering\includegraphics[scale=0.4]{imagenes/explotacion/5.jpg}
	\caption{Paso 4: lista de objetos móviles}
	\label{paso4ListaObjetosMoviles}
\end{figure}

\section{Simulación y evaluación de un sistema de recomendación}

\subsection{Simulación de un escenario realista}

Una vez que hayamos realizado la configuración del escenarios realista vamos a realizar simulaciones con dos usuarios y recoger las valoraciones que han realizado para su posterior evaluación. En la figura \ref{simulacionYEvaluacion} podemos ver una captura de una simulación realizada:

\begin{figure}[H]
	\centering\includegraphics[scale=0.35]{imagenes/explotacion/6.jpg}
	\caption{Simulación realizada en San Luis Potosí, México}
	\label{simulacionYEvaluacion}
\end{figure}

\subsection{Evaluación de un recomendador}




\section{Rendimiento}



%w
\chapter{Gestión del proyecto}

En este capítulo se explican el modelo de proceso seleccionado y las distintas etapas por las que ha pasado el este Trabajo Fin de Grado centradose únicamente en los aspectos más importantes.

\section{Modelo de proceso seleccionado}


Este Trabajo Fin de Grado ha surgido una importante evolución desde el primer planteamiento hasta la obtención del sistema final. La primera idea para la evaluación de los algoritmos de recomendaciones era la utilización de un videojuego ya que este podría recolectar datos de forma transparente mientras los usuarios se divertian. El videojuego consistiría en el cumplimiento de misiones en una cuidad basandose en algún videojuego de aventuras como el videojuego Paperboy del año 1985 donde un chico reparte periódicos en una cuidad.

Por esto se ha planteado usar el motor gráfico Unity 5 para el desarrollo del videojuego ya que no tenía sentido desarrollar un videojuego usando simplemente un lenguaje de programación. Pero existía la incertidumbre si todos los requisitos podrían ser implementados. Esto era debido porque por una parte no se conocían de antemano todos los requisitos y por otra no se conocía si el motor gráfico tenía algún tipo de limite. 

Para solucionar estos problemas se ha tomado la decisión de elegir el modelo en espiral como modelo de trabajo. Las actividades de este modelo forman una espiral de tal forma que cada iteración representa un conjunto de actividades. Esto nos permitiría segmentar el trabajo en tareas más pequeñas e ir definiendo los requisitos mientras se desarrollaba el videojuego. De esta manera podríamos evaluar si los requisitos propuestos podrían o no ser implementados con Unity 5. En caso de que el motor gráfico tuviese algún límite tendríamos la capacidad de reorientar el proyecto.

\subsection{Primer ciclo de la espiral: formación con las herramientas a utilizar}

En la tabla \ref{tabla:requisitosEtapa1} podemos encontrar las tareas que han sido definidas para esta etapa:

\begin{table}[H]
\begin{center}
\begin{tabular}{|p{1.5cm}| p{10.5cm}|}
\hline 
Código & Descripción \\
\hline \hline
RQ-0  & Formación básica con Unity 5\\ \hline
RQ-1  & Establecer puntos en el mapa para que el jugador pueda ir a buscarlos. \\ \hline
RQ-2  & Los puntos establecidos en el requisitos RQ-1 deben de ser extraibles desde un servidor externo\\ \hline
RQ-3  & Establecer una conexión HTTP/JSON entre Unity 5 y un servidor externo. \\ \hline
RQ-4  & Sacar el modelado geométrico de un servidor externo \\ \hline
RQ-5  & El videojugo debe de tener un menú principal \\ \hline
RQ-6  & Investigar si se puede usar Longitud y Latitud con Unity 5 \\ \hline
RQ-7  & Investigar si los edificio generados con CityEngine 2014 son reales o no \\ \hline
RQ-8  & Permitir que el juego se realice sobre mapas de ciudades reales \\ \hline
RQ-9  & Investigar si es posible el desarrollo de videojuego en 3D \\ \hline
\end{tabular}
\caption{Tareas del primer ciclo de la espiral}
\label{tabla:requisitosEtapa1}
\end{center}
\end{table}

Como conclusión de esta etapa obtenemos el siguiente resultado:
\begin{itemize}
	\item puede establecerse una conexión con un servidor externo mediante HTTP/JSON (RQ-3). 
	\item pueden establecerse puntos en el mapa para que el jugador pueda ir a buscarlos(RQ-1). 
	\item pueden extraerse puntos desde un servidor externo para que el usuarios pueda ir a buscarlos (RQ-1 y RQ-3).
	\item la realizacion del menú principal del videojuego ha sido posible.
	\item Unity 5 utiliza eje de abscisas. Por lo tanto si queremos usar coordenadas geográficas tenemos que calcular la Longitud y Latitud a partir del eje de coordenadas (x, y, z).
	\item los datos sobre los edificios de OpenStreetMap no están completos.
	\item se ha decido de dejar atrás el modelado 3D ya que se han detectado los siguientes problemas: coste en cuanto a tiempo demasiado grande para el diseño de una ciudad, CityEngine 2014 no interpreta correctamente los datos exportados además no existe ninguna otra herramienta que nos permita la correcta interpretación de los datos (por lo menos yo no podido encontrarla).
\end{itemize}

\subsection{Segundo ciclo de la espiral: diseñar el mapa y desarrollar la IA}

En la tabla \ref{tabla:requisitosEtapa2} podemos encontrar las tareas que han sido definidas para esta etapa:

\begin{table}[H]
\begin{center}
\begin{tabular}{|p{1.5cm}| p{10.5cm}|}
\hline 
Código & Descripción \\
\hline \hline
RQ-10 & Investigar como se desarrollan grafos en Unity para la posterior implementación de la IA sobre estos para el movimiento de los vehículos etc. \\ \hline
RQ-11 & Diseñar y desarrollar un mapa del juego\\ \hline
\end{tabular}
\caption{Tareas segundo ciclo de la espiral}
\label{tabla:requisitosEtapa2}
\end{center}
\end{table}

Como conclusión de esta etapa obtenemos el siguiente resultado:
\begin{itemize}
	\item Unity 5 dispone de su propio modulo de IA y por lo tanto no hace falta implementar algoritmos de IA para el comportamiento de los objetos dinámicos.
	\item en esta etapa se ha definido el requisito que el juego tiene que funcionar sobre cualquier mapa del mundo. Por lo tanto en la siguiente etapa hay que investigar si es posible la integración de Unity 5 con OpenStreetMap.
\end{itemize}

\subsection{Tercer ciclo de la espiral: integración con OpenStreetMap}

En la tabla \ref{tabla:requisitosEtapa3} podemos encontrar las tareas que han sido definidas para esta etapa:

\begin{table}[H]
\begin{center}
\begin{tabular}{|p{1.5cm}| p{10.5cm}|}
\hline 
Código & Descripción \\
\hline \hline
RQ-11 & Investigar en que consiste el formato OSM\\ \hline
RQ-12 & Investigar si es posible la integración de OpenStreetMap con Unity\\ \hline
\end{tabular}
\caption{Tareas tercer ciclo de la espiral}
\label{tabla:requisitosEtapa3}
\end{center}
\end{table}

Como conclusión de esta etapa obtenemos el siguiente resultado:
\begin{itemize}
	\item Unity 5 no puede ser integrado con OpenStreetMap y no es capaz de renderizar mapas a partir sus datos (ficheros OSM). 
	\item en esta etapa se ha decido reorientar el proyecto por las siguientes razones: Unity 5 no puede ser integrado con OpenStreetMap por lo tanto la única opción para desarrollar un videojuego sobre mapas reales es desarrollando el videojuego desde cero con algún lenguaje como Java y el inconveniente que esto representa es que es demasiado costoso en cuanto a tiempo y dificultad.  
\end{itemize}

\subsection{Cuarto ciclo de la espiral: instalación y configuración del entorno de trabajo para el desarrollo del simulador RecSim}

En la tabla \ref{tabla:requisitosEtapa4} podemos encontrar las tareas que han sido definidas para esta etapa:

\begin{table}[H]
\begin{center}
\begin{tabular}{|p{1.5cm}| p{10.5cm}|}
\hline 
Código & Descripción \\
\hline \hline
RQ-12 & instalar mongoDB\\ \hline
RQ-13 & instalar Node.js y NPM\\ \hline
RQ-14 & instalar git\\ \hline
RQ-14 & instalar Java y Apache maven\\ \hline
RQ-16 & Desarrollar y configurar la base de la aplicación con Node.js\\ \hline
RQ-17 & Instalar y configurar Angular.js\\ \hline
RQ-18 & Instalar y configurar OpenLayers.js\\ \hline
RQ-19 & Instalar y configurar Bootstrap\\ \hline
\end{tabular}
\caption{Tareas cuarto ciclo de la espiral}
\label{tabla:requisitosEtapa4}
\end{center}
\end{table}

Como conclusión de esta etapa obtenemos el siguiente resultado:
\begin{itemize}
	\item se han instalado y configurado todas las herramientas y frameworks necesarios para el desarrollo de la aplicación.
\end{itemize}

\subsection{Quinto ciclo de la espiral: desarrollo del simulador RecSim}

En la tabla \ref{tabla:requisitosEtapa5} podemos encontrar las tareas que han sido definidas para esta etapa:

\begin{table}[H]
\begin{center}
\begin{tabular}{|p{1.5cm}| p{10.5cm}|}
\hline 
Código & Descripción \\
\hline \hline
RQ-20 & Definir lo requisitos funcionales y no funcionales de la aplicación\\ \hline
RQ-21 & Desarrollar el front-end\\ \hline
RQ-22 & Deseñar la base de datos\\ \hline
RQ-23 & Diseñar y desarrollar el back-end\\ \hline
RQ-24 & Diseñar y desarrollar el recomendador\\ \hline
RQ-25 & Realizar pruebas funcionales de la aplicación\\ \hline
\end{tabular}
\caption{Tareas quinto ciclo de la espiral}
\label{tabla:requisitosEtapa5}
\end{center}
\end{table}

Como conclusión de esta etapa obtenemos el siguiente resultado:
\begin{itemize}
	\item la parte mas costosa de esta etapa ha sido la definición de los requisitos de la aplicación.
	\item una vez definidos los requisitos el resto del trabajo ha sido puramente técnico. 
\end{itemize}


\subsection{Sexto ciclo de la espiral: documentar el trabajo realizado}

La última etapa consiste en documentar el trabajo realizado en este Trabajo Fin de Grado.

\newpage

\section{Tiempo dedicado}

Como ya se ha comentado anteriormente, el desarrollo del proyecto se ha realizado siguiendo el modelo en espiral. En esta sección se mostrará el tiempo dedicado y también un cronograma de las diferentes etapas.

\begin{table}[H]
\begin{center}
\begin{tabular}{|c|c|}
\hline 
\noindent{\textbf{Fase}} & \noindent{\textbf{Horas}} \\
\hline \hline
Fase 1 & 54\\ \hline
Fase 2 & 6.5\\ \hline
Fase 3 & 38\\ \hline
Fase 4 & 3.5\\ \hline
Fase 5 & 247\\ \hline
Fase 6 & 42.5\\ \hline
Reuniones & 7\\ \hline
Pruebas de carga & 11.5 \\ \hline
\hline \hline
\noindent{\textbf{Total}} & \noindent{\textbf{410}} \\ \hline
\end{tabular}
\caption{Separación por horas de las distintas fases}
\label{tabla:requisitosEtapa5}
\end{center}
\end{table}
 

\begin{figure}[H]
\centering\includegraphics[scale=0.8]{imagenes/gantt.jpg}
\caption{Diagrama de Gantt de las distintas fases del proyecto}
\label{gantt}
\end{figure}

%
\input{capitulos/05_Conclusiones}

\nocite{*}
\bibliography{bibliografia/bibliografia}\addcontentsline{toc}{chapter}{Bibliografía}
\bibliographystyle{unsrt}

\renewcommand{\appendixname}{Anexos}
\renewcommand{\appendixtocname}{Anexos}
\renewcommand{\appendixpagename}{Anexos}

\appendix
\clearpage
\addappheadtotoc
\appendixpage

\chapter{Manual del usuario}

\section{Visión general}

\subsection{Introducción}

En el siguiente documento se recoge una descripción del funcionamiento del simulador de escenarios con usuarios móviles para la evaluación de algoritmos de recomendaciones.

\subsection{¿Qué es el simulador de escenarios con usuarios móviles?}

El simulador de escenarios con usuarios móviles es una herramienta que trata de simular distintos tipos de algoritmos de recomendaciones en el entorno de una ciudad real con el fin de evaluar su correcto funcionamiento.

Se trata de un sistema multiusuario donde distintos tipos de usuarios se conectan y se mueven en una ciudad real (el entorno es configurado de antelación). El recomendador tiene en cuenta distintos tipos de parámetros: desde sus posiciones geográficas hasta sus perfiles y preferencias.

\subsection{Tipos de tecnologías utilizadas}

Esta herramienta consta de dos partes: la primera es el simulador y la segunda es el recomendador.

El Simulador esta desarrollado con nodejs, sockets-io, angularjs, bootstrap y como base de datos utiliza mongodb. El recomendador está desarrollado con java. La integración entre el navegador (cliente), simulador y recomendador está realizada mediante un sistema de evetos bidireccionales (sockets-io). De esta forma conseguimos comunicar todas la partes del sistema en tiempo real.

\subsection{Instalación}

\subsubsection{Paso 1: Instalar mongoDB}

Lo primero que tenemos que hacer es instalar mongodb. Los pasos para la instalación de mongodb depeden del tipo de sistema operativo que disponemos. Por esto no vamos a entrar en detalle de como se instala y vamos a seguir el tutorial disponible en la web oficial:

\begin{itemize}
	\item Linux: https://docs.mongodb.org/manual/administration/install-on-linux/
	\item Windows: https://docs.mongodb.org/manual/tutorial/install-mongodb-on-windows/
	\item Mac OS: https://docs.mongodb.org/manual/tutorial/install-mongodb-on-os-x/
\end{itemize}

\subsubsection{Paso 2: Instalar Node.js y NPM}

Vamos en la web oficial de nodejs (https://nodejs.org) y descargamos e instalamos la version v0.12.4. En el caso de Windows la instalación es igual que la de cualquier otro programa. Para la instalación en otros SO visitar https://nodejs.org/en/download/.

A continuación tenemos que instalar el gestor de paquetes NPM v2.10.1. En el caso de Windows vamos en la web oficial (https://nodejs.org/en/download/) y nos descargamos e instalamos el ejecutable. En el caso de Linux ejecutamos el siguiente comando en la consola:
\begin{lstlisting}[language=xml, frame=single]
sudo apt-get install npm
\end{lstlisting}

\subsubsection{Paso 3: Instalar git}

Git es un sistema distribuido de control de versiones. Para la instalación de este nos descargamos el ejecutable de https://git-scm.com/downloads y seguimos los pasos que nos indica este.

\subsubsection{Paso 4: Clonar el proyecto de github e instalarlo}

A continuacion tenemos que clonar el proyecto del Simulador de github. Para esto abrimos una consola y nos situamos en el directorio donde queremos clonar el proyecto. A continuacion ejecutamos el siguiente comando:
\begin{lstlisting}[language=xml, frame=single]
git clone https://github.com/slavcho87/Simulator
\end{lstlisting}
Vemos que se ha creado un directorio llamado Simulator. Lo primero que tenemos que hacer es bajarnos todas las dependencias del proyecto. Por esto ejecutamos el siguiente comando:
\begin{lstlisting}[language=xml, frame=single]
npm install
\end{lstlisting}
Una vez que nos hemos clonado el proyecto y descargado las dependencias de este podemos arrancar el servidor mediante el siguiente comando:
\begin{lstlisting}[language=xml, frame=single]
npm start
\end{lstlisting}

\paragraph{Paso 4.1: Configuraciones básicas del simulador}

En el fichero baseConfig.json disponemos de las siguientes basicas para el simulador como el puerto donde se ejecuta el servidor y la localizacion de la base de datos. Si editamos el fichero baseConfig.json veremos que tiene el siguiente contenido:
\begin{lstlisting}[language=xml, frame=single]
{ 
"port": 81, 
"locationDB": "localhost", 
"nameDB": "simulator" 
}
\end{lstlisting}

Las variables del fichero baseConfig.json tienen el siguiente significado:

\begin{itemize}
	\item port: es el puerto donde se va a ejecutar el servidor
	\item locationDB: es la localizacion donde se va a ejecutar mongodb.
	\item nameDB: es el nombre del esquema de la base de datos donde nos conectamos.
\end{itemize}

Dichas configuraciones son importantes ya que de esta forma tenemos la opción de llevarnos la base de datos en un servidor diferente para darle más potencia.

\paragraph{Paso 4.2: Posibles problemas durante la ejecución del simulador}

Se puede dar el caso que al intentar arrancar el simulador nos de el siguiente error:

\begin{lstlisting}[language=xml, frame=single]
Error: listen EACCES 0.0.0.0:81
at Object.exports._errnoException (util.js:870:11)
at exports._exceptionWithHostPort (util.js:893:20)
at Server._listen2 (net.js:1218:19)
at listen (net.js:1267:10)
at Server.listen (net.js:1363:5)
\end{lstlisting}

El error EACCES ocurre cuando no tenemos suficientes privilegios sobre el puerto donde estamos lanzando el servidor Node.js. La solución depende del Sistema Operativo que estamos usando. En el caso de Windows tenemos que cambiar el puerto donde lanzamos el servidor siempre y cuando estemos usando un usuario que tenga suficientes privilegios. En el caso de Linux tenemos que lanzar el servidor con el comando sudo delante de la siguiente manera:

\begin{lstlisting}[language=xml, frame=single]
slavcho@ubuntu:~/Simulator/scripts$ sudo ./ejecutarSimulador.sh
\end{lstlisting}

\subsubsection{Paso 5: Instalación del recomendador}

\paragraph{Paso 5.1: Instalar Apache maven}

Antes de todo tenemos que instalar Apache maven que es un gestor de paquetes. Por lo tanto vamos en la web oficial (https://maven.apache.org/) y descargamos y descomprimimos el ficheros compromido. A continuación tenemos que añadir en la variable PATH la dirección de la carpeta donde hemos descomprimido maven. Para ver que maven se haya instalado correctamente ejecutamos el siguiente comando:
\begin{lstlisting}[language=xml, frame=single]
mvn -v
\end{lstlisting}
De esta forma comprobamos la version de maven que tenemos instalado. Tenemos que ver una salida como la siguiente: \newline

\begin{lstlisting}[language=xml, frame=single]
Apache Maven 3.2.5 (12a6b3acb947671f09b81f49094c53f426d8cea1; 2014-12-14T18:29:23+01:00)
Maven home: C:\maven
Java version: 1.7.0_79, vendor: Oracle Corporation
Java home: C:\Program Files (x86)\Java\jdk1.7.0_79\jre
Default locale: es_ES, platform encoding: Cp1252
OS name: "windows 8.1", version: "6.3", arch: "x86", family: "windows"
\end{lstlisting}

\paragraph{Paso 5.2: Compilar el recomendador}

Una vez que hayamos instalado maven correctamente tenemos que compilar el recomendador. Por esto ejecutamos el ficheros compilarRecommender que se encuentra en la carpeta scripts. Existen dos versiones: uno para Windows y otro para Linux.

\paragraph{Paso 5.3: Configuraciones básicas del recomendador}

Si editamos el fichero configs/baseConfig.txt podemos ver que tiene el siguiente formato:
\begin{lstlisting}[language=xml, frame=single]
<config>
<host>http://localhost</host>
<port>81</port> 
<hostMongo>localhost</hostMongo>
<portMongo>27017</portMongo> 
<nameDB>simulator</nameDB> 
</config>
\end{lstlisting}

\begin{itemize}
	\item host: es la dirección donde se ejecuta el simulador. En el ejemplo este se está ejecutando en local
	\item port: es el puerto donde se ejecuta el simulador. En el ejemplo este se ejecuta en el puerto 81
	\item hostMongo: es la dirección de la base de datos. En el ejemplo esta se está ejecutando en local
	\item portMongo: es el puerto donde se ejecuta la base de datos.
	\item nameDB: es el nombre del esquema de la base de datos
\end{itemize}

\paragraph{Paso 5.4: Ejecutar el recomendador}

Para ejecutar el recomendar tenemos que ejecutar el ficheros ejecutarRecommender que se encuentra en la carpeta scripts. Existe dos versiones de este fichero: uno para Windows y otro para Linux.

\subsection{Primeros pasos}

El primero paso al instalar el simulador es crear nuestro usuario (apartado \ref{sec:CrearUsuario}). A continuación vamos en Settings y realizamos las configuraciones realizadas en los capítulos \ref{sec:confRecomendador}, \ref{sec:confObjEstaticos} y \ref*{sec:confObjDinamicos}.

Se trata de crear nuevas configuraciones para el recomendador y crear los tipos de objetos dinámicos y estáticos que usaremos posteriormente en la creación de mapa (apartado \ref{sec:crearMapa}) y escenas (apartado \ref{sec:crearEscena}).

\section{Configuración del recomendador}\label{sec:confRecomendador}

Para configurar los parámetros del recomendador primero tenemos que estar autentificados con nuestros usuario. Una vez autentificados vamos en Settings $\rightarrow$ Recommender settings y vemos la siguiente pantalla:

\begin{figure}[H]
	\centering\includegraphics[scale=0.35]{imagenes/capitulo2/configuracion-recomendador.jpg}
	\caption{Configuración del recomendador}
	\label{img:ConfiguracionRecomendador}
\end{figure}


En este formulario tenemos que introducir los siguientes datos:
\begin{itemize}
	\item Pool name: es nombre que vamos a dato al conjunto de parámetros
	\item Recommender type: es el tipo de recomendador. Podemos elegir entre pull (el usuario solicita una recomendación) o push (el recomendador realiza recomendaciones sin que el usuario lo haya solicitado)
	\item Strategy type:: indicar el tipo de implementación que queremos para el recomendador
	\item Maximum distance to go (meters): es la distancia máxima (en metros) que está dispuesto a recorrer el usuario
	\item Visibility radius (meters): radio (en metros) de visibilidad del usuario. Los ítems que están fuera de este radio son invisibles para el usuario 
	\item Number of ítems to recommend: número máximo de ítems que va a recomendar el recomendador cada vez
	\item Minimum score for recommending an ítem: puntuación mínima para que un ítem sea recomendado
\end{itemize}

\newpage

\section{Configuración de los objetos estáticos}\label{sec:confObjEstaticos}

Para crear un nuevo tipo de objeto estático primero tenemos que estar autentificados con nuestros usuario. Una vez autentificados vamos en Settings $\rightarrow$ Static ítem type y vemos la siguiente pantalla:

\begin{figure}[H]
	\centering\includegraphics[scale=0.35]{imagenes/capitulo3/config-objetos-estaticos.jpg}
	\caption{Configuración de los objetos estáticos}
	\label{img:ConfiguracionObjetosEstaticos}
\end{figure}

En este formulario tenemos que introducir los siguientes datos:

\begin{itemize}
	\item Name: es el nombre del tipo del objeto estático
	\item Icon: icono del tipo del objeto estático
\end{itemize}	

Por debajo del formulario de creación nuevos tipos de objetos estáticos aparecerá la lista con todos los tipos de objetos estáticos que han sido creados. En el apartado opciones podemos gestionar cada uno de estos objeto.

\section{Configuración de los objetos dinámicos}\label{sec:confObjDinamicos}

Para crear un nuevo tipo de objeto dinámicos primero tenemos que estar autentificados con nuestros usuario. Una vez autentificados vamos en Settings $\rightarrow$ Dynamic ítem type y vemos la siguiente pantalla:

\begin{figure}[H]
	\centering\includegraphics[scale=0.35]{imagenes/capitulo4/config-objetos-dinamicos.jpg}
	\caption{Configuración de los objetos dinámicos}
	\label{img:ConfiguracionObjetosDinamicos}
\end{figure}

En este formulario tenemos que introducir los siguientes datos:

\begin{itemize}
	\item Name: es el nombre del tipo del objeto dinámicos
	\item Icon: icono del tipo del objeto dinámicos
\end{itemize}

Por debajo del formulario de creación nuevos tipos de objetos dinámicos aparecerá la lista con todos los tipos de objetos dinámicos que han sido creados. En el apartado opciones podemos gestionar cada uno de estos objeto.

\section{Crear un un nuevo usuario}\label{sec:CrearUsuario}

Para registrar un nuevo usuario lo que tenemos que hacer es pinchar en el enlace Register en la página lo login. Se nos muestra la siguiente pantalla:

\begin{figure}[H]
	\centering\includegraphics[scale=0.25]{imagenes/capitulo5/register.jpg}
	\caption{Crear un nuevo usuario}
	\label{img:AddUser}
\end{figure}

En este formulario tenemos que añadir el nombre del usuario, su contraseña y subir su imagen. Esta imagen aparecerá en el mapa del simulador. Una vez que hayamos creado el usuario tenemos que pulsar el botón "Go to login" para volver en la pantalla de login para introducir el nombre de nuestro usuario su contraseña.

\section{Actualizar el perfil de un usuario}

Para realizar cambios en el nombre del usuario, cambiar la imagen del perfil o cambiar la contraseña tenemos que ir en el menú Perfil:

\begin{figure}[H]
	\centering\includegraphics[scale=0.35]{imagenes/capitulo6/perfil-del-usuario.jpg}
	\caption{Actualizar el perfil de un usuario}
	\label{img:UpdateUser}
\end{figure}

Vemos que hay 3 formularios:

\begin{itemize}
	\item uno para cambiar el nombre del usuario
	\item otro para cambiar la contraseña del usuario
	\item otro para cambiar la imagen del perfil del usuarios
\end{itemize}

Al realizar cualquier cambio el el sistema nos informara si la acción ha salido bien o no.

\section{Búsqueda de mapas}\label{sec:BuscarMapas}

La opción Maps $\rightarrow$ map search nos permite buscar mapas a los cuales queremos conectarnos para realizar alguna simulación:

\begin{figure}[H]
	\centering\includegraphics[scale=0.35]{imagenes/capitulo7/busqueda-de-mapas.jpg}
	\caption{Búsqueda de mapas}
	\label{img:BuscarMapas}
\end{figure}

En la imagen vemos que disponemos de distintos tipos de filtros para la búsqueda de mapas. Estos filtros son: por nombre, por tipo, por estado, por ciudad, por fechas de creación o solo buscar solo mis mapas. Si no introducimos ningún filtro devolverá todos los mapas que están creados en el sistema.

\section{Crear un nuevo mapa}\label{sec:crearMapa}

Para crear un nuevo mapa vamos en Maps $\rightarrow$ new map y vemos el siguiente formulario:

\begin{figure}[H]
	\centering\includegraphics[scale=0.35]{imagenes/capitulo8/crear-un-nuevo-mapa.jpg}
	\caption{Crear un nuevo mapa}
	\label{img:AddMapa}
\end{figure}

Tenemos que rellenar los siguientes datos:

\begin{itemize}
	\item Name: es el nombre que queremos dar al mapa.
	\item Type: tipo que mapa. Podemos elegir entre publico (puede conectarse cualquier usuarios) y privado (solo puede conectarse el que la ha creado).
	\item State: estado del mapa. Podemos elegir entre activa (que el mapa está disponible para realizar simulaciones) y borrador (el mapa todavía no está disponible para realizar simulaciones)
\end{itemize}

Una vez que hayamos rellenado el formulario pulsamos en el botón Save y el sistema nos informará si el guardado ha salido con éxito o no. Si se ha guardado con éxito por debajo de este formulario aparece el formulario de creaciones de escenas y una lista de las escenas actuales. Lo normales es que la lista de escenas creadas aparezca vacía ya que todavía no hemos creado ninguna escena. Para ver los detalles de como crear una escena consultar el capítulo \ref{sec:crearEscena}.

\section{Crear una nueva escena}\label{sec:crearEscena}
Una vez que hayamos creado el mapa podemos empezar a crear las escenas. La creación de escenas se realiza en 4 pasos que vamos a ver a continuación.

\subsection{Paso 1: Elegir el nombre de la escena y el recomendador a utilizar}

En el primer paso tenemos que escoge un nombre para la escena que estamos configurando y elegir el tipo de recomendador que queremos usar el esta escena.

\begin{figure}[H]
	\centering\includegraphics[scale=0.35]{imagenes/capitulo9/crear-escena-1.JPG}
	\caption{Paso 1: elegir nombre de la escena y el recomendador}
	\label{img:paso1}
\end{figure}

\subsection{Paso 2: Límites de la escena}

En este paso tenemos que buscar la ciudad donde se va a realizar la simulación y cuales son los límites de la escena.

\begin{figure}[H]
	\centering\includegraphics[scale=0.35]{imagenes/capitulo9/crear-escena-2.JPG}
	\caption{Paso 2: Limites de la escena}
	\label{img:paso2}
\end{figure}

Para buscar la cuidad donde se va a realizar la simulación tenemos poner cual es su nombre en la casilla City y pulsar el botón Go!. Vemos que al lado nos aparece un desplegable con una lista de resultados. Elegimos uno de ellos y vemos que el mapa que aparece al lado se actualiza.

Para definir los límites de la escena tenemos que establece cual es la esquina superior izquierda y la esquina inferior derecha. Para definir la esquina superior izquierda seleccionamos el select button select upper left corner y hacemos click en el mapa para definir donde estará dicha esquina. Veremos que las cajas de texto Upper left corner: latitude y Upper left corner: longitude aparecerán las coordenadas geográficas de dicha esquina (longitud y latitud).

Para definir la esquina inferior derecha seleccionamos la opción select lower right corner y hacemos click en el mapa para definir donde estará dicha esquina. Cuando hayamos seleccionado la esquina inferior derecha vemos que en las cajas de texto Lower right corner: latitude y Lower right corner: longitude aparecerán las coordenadas geográficas de dicha escena.


\subsection{Paso 3: Configurar los objetos estáticos}

Este paso consiste en definir cuales son los objetos estáticos y sus posiciones. Para definirlos disponemos de dos opciones: importarlos desde un fichero con formato JSON y deninirlos manualmente.

\subsubsection{Paso 3.1: importar los objetos estáticos}

Para importa los objetos estáticos seleccionamos la opciones load from file y nos aparecerá un formulario donde podemos elegir cual es el fichero que queremos importar. Antes de que se carguen los datos se nos previsualiza el contenido del fichero que queremos cargar:

\begin{figure}[H]
	\centering\includegraphics[scale=0.35]{imagenes/capitulo9/crear-escena-3-1.JPG}
	\caption{Paso 3.1: Previsualizar los objetos estáticos a importar}
	\label{img:paso3-1-1}
\end{figure}

En esta ventana tenemos que vincular los atributos del fichero a los atributos de nuestro sistema. Para cada atributo de nuestro sistema tenemos un select y en este select salen los atributos del fichero. Tenemos que seleccionar que atributo del fichero a que atributo de nuestro sistema queremos vincular. Por ultimo tenemos que elegir que tipo de objeto estático queremos asignar a los datos que estamos importando. Una vez importados los objetos nos aparecerá un table con todos los objetos importados.

\begin{figure}[H]
	\centering\includegraphics[scale=0.35]{imagenes/capitulo9/crear-escena-3.JPG}
	\caption{Paso 3.1: importar los objetos estáticos desde un fichero}
	\label{img:paso3-1}
\end{figure}

Una vez importados los objetos vemos que disponemos una serie de opciones que nos permiten gestionar los objetos importados. Podemos seleccionar un subconjunto y todos los objetos y cambiarles el tipo o borrarles. Esto se hace con las opciones que tenemos en la cabecera de la tabla.

\subsubsection{Paso 3.2: definir los objetos estáticos manualmente}

Para definirlos de forma manual tenemos que seleccionar la opción set manually y nos aparecer un formulario con el tipo de ítem que queremos definir, el nombre que queremos asignarle. Para definir cuales son sus coordenadas geográficas tenemos que seleccionar la posiciones directamente sobre el mapa. A continuación pulsamos sobre el botón Save Item y nos aparecerá la misma tabla que el en paso 3.1:

\begin{figure}[H]
	\centering\includegraphics[scale=0.35]{imagenes/capitulo9/crear-escena-4.JPG}
	\caption{Paso 3.2:  definir los objetos estáticos manualmente}
	\label{img:paso3-2}
\end{figure}

\subsection{Paso 4: Configurar los objetos dinámicos}

El últimos paso consiste en definir los objetos dinámicos y sus rutas. Disponemos de 3 posibles maneras para definirlos: importándolos desde un fichero con formato JSON, definirlos manualmente, generarlos aleatoriamente.

\subsubsection{Paso 4.1: importar los objetos dinámicos}

Para importarlos objetos dinámicos desde un fichero tenemos que seleccionar la opción load from file y nos aparecerá un formulario que nos permite seleccionar el fichero a importar. Antes de cargar los objetos dinámicos podemos previsualizar los datos y vincular los atributos del fichero a los atributos del sistema. Esto se realiza mediante los selects que observamos en la siguiente pantalla:

\begin{figure}[H]
	\centering\includegraphics[scale=0.35]{imagenes/capitulo9/crear-escena-8.JPG}
	\caption{Paso 4.1: importar los objetos dinámicos}
	\label{img:paso4-1}
\end{figure}

Una vez que hayamos importado los ficheros vemos una tabla parecida que tabla del paso 3.1. La diferencia es que aquí disponemos de más opciones. Entre estas opciones son las de generar las rutas a un subconjunto a todos los datos, cambiar el tipo de objeto dinámico y borrarlos objetos seleccionados.

\subsubsection{Paso 4.2: definir los objetos dinámicos manualmente}

Para definir los objetos dinámicos manualmente seleccionamos la opción set manually nos aparecer un formulario en el cual tenemos que elegir el tipo de objeto dinámico, el nombre que queremos asignarle y su velocidad. Para definir la trayectoria de este tenemos que seleccionar los nodos del grafo de movimiento directamente en el mapa y nos aparecerá una tabla que contiene dichos nodos:

\begin{figure}[H]
	\centering\includegraphics[scale=0.35]{imagenes/capitulo9/crear-escena-6.JPG}
	\caption{Paso 4.2: definir los objetos dinámicos manualmente}
	\label{img:paso4-2-1}
\end{figure}

Una vez que hayamos definido el trayecto pulsamos sobre el botón Save ítem y nos saldrá la lista con todos los objetos dinámicos en la escena:

\begin{figure}[H]
	\centering\includegraphics[scale=0.35]{imagenes/capitulo9/crear-escena-7.JPG}
	\caption{Paso 4.2: definir los objetos dinámicos manualmente}
	\label{img:paso4-2-2}
\end{figure}

\subsubsection{Paso 4.3: generar los objetos dinámicos aleatoriamente}

Para generar los trayectos de forma aleatoria tenemos que seleccionar la opción Random trajectory y aparecerá un formulario donde tenemos que indicar el tipo de objeto dinámico que queremos generar, la cantidad de objetos que queremos generar y el tipo de vía en la que queremos que aparezcan. A continuación pulsamos sobre el botón Generate random way y tenemos que esperar que se generen las rutas. Este proceso puede ser bastante costoso y tenemos que ser pacientes. Esto es debido porque se baja el grafo de la escena y este pesa algunos cuantos megas.

\begin{figure}[H]
	\centering\includegraphics[scale=0.35]{imagenes/capitulo9/crear-escena-5.JPG}
	\caption{Paso 4.3: generar los objetos dinámicos aleatoriamente}
	\label{img:paso4-3}
\end{figure}

\section{Edición de mapas y escenas}\label{sec:editarMapasEscenas}

Para modificar un mapa o escena primero tenemos que realizar una búsqueda de mapas en Maps $\rightarrow$ maps search y en la lista de resultados pulsamos sobre el icono de editar imagen. Solo podemos editar un mapa si la hayamos creado nosotros. De lo contrario el icono de editar imagen no aparecerá. Una vez que hayamos pulsado el icono de editar imagen entonces veremos la siguiente pantalla:

\begin{figure}[H]
	\centering\includegraphics[scale=0.35]{imagenes/capitulo10/capitulo10.jpg}
	\caption{Edición de mapas y escenas}
	\label{img:UpdateMapScene}
\end{figure}

En esta pantalla disponemos de distintos tipos de opciones entre los cuales modificar datos del mapa, crear o modificar escenas.

\section{Simulación}

Para ejecutar una simulación lo primero que tenemos que hacer es realizar la búsqueda de un mapa explicado en el capítulo \ref{sec:BuscarMapas}. Una vez que hayamos realizado la búsqueda pulsamos el botón ``Play`` y a continuación se nos muestra una pantalla en la cual tenemos que elegir la escena a la cual queremos conectarnos.

\begin{figure}[H]
	\centering\includegraphics[scale=0.35]{imagenes/capitulo11/capitulo11.jpg}
	\caption{Simulación}
	\label{img:Simulation}
\end{figure}

Para iniciar la simulación y que todos los ítems empiecen a moverse tenemos que pulsar el botón play. Hay que tener en cuenta que existe la posibilidad que de ya haya usuario que estén ejecutando una simulación sobre este escenario. Por esto cuando nos conectemos a la escena veremos que los objetos ya se estén moviendo. En cualquier momento podemos pausar la simulación. Entonces la simulación se pausa en todos los usuarios que están conectados en la escena. En cualquier momento cualquier usuario puede reanudar la simulación. Para salir de la simulación lo único que tenemos que hacer es salir de la pantalla de simulación.

\subsection{Controles del usuario}

Una vez cargada la escena podemos elegir entre usar la posición GPS del dispositivo con el cual nos estamos conectado y simular los movimientos del usuario.

\subsubsection{Coordenadas GPS}

Para usar el posicionamiento por GPS tenemos que elegir GPS positioning en el select con nombre User movement y nuestro icono aparecerá en el mapa. A medida que nos vamos moviendo vemos que nuestra posición se va actualizando.

\subsubsection{Movimiento simulado}

Para usar el movimiento simulado tenemos que elegir Simulated movement en el select con nombre User movement y a continuación tenemos que elegir donde queremos situarnos en el mapa haciendo click en el mapa. A continuación podemos empezar a movernos por el mapa con los siguientes controles:

\begin{itemize}
	\item tecla w: movimiento hacia arriba
	\item tecla s: movimiento hacia abajo
	\item tecla a: movimiento hacia la izquierda
	\item tecla d: movimiento hacia la derecha
	\item espacio: pausar el movimiento del usuario
\end{itemize}

\subsection{Recomendaciones}

Para obtener recomendaciones tenemos que seguir los siguientes pasos:

\begin{itemize}
	\item paso 1: seleccionamos los tipos de ítems sobre los cuales queremos obtener resultados (ítems types to recommend)
	\item paso 2: en Result type show elegimos como queremos que se nos muestren los resultados. Por defecto si el recomendador no tiene items que recomendar mostrará una lista vacía.
	\item paso 3: pulsamos el botón Recommend ítems para obtener resultados
\end{itemize}

\subsection{Votaciones}

Para realizar las votaciones disponemos de los siguientes opciones:

\begin{itemize}
	\item opción 1: pulsamos sobre el ítem concreto en el mapa y se nos abrirá un desplegable donde podemos votar
	\item opción 2: una vez que hayamos obtenido una lista con ítems recomendados disponemos de un desplegable donde podemos realizar la votación
\end{itemize}

\newpage

\section{Evaluación de un recomendador}

Para realizar la evaluación de un recomendador vamos en el menú Evaluations:

\begin{figure}[H]
	\centering\includegraphics[scale=0.35]{imagenes/capitulo12/evaluacion.png}
	\caption{Evaluación de un recomendador}
	\label{img:evaluacion}
\end{figure}

Lo primero que tenemos que hacer es seleccionar el tipo de recomendador que queremos evaluar. Una vez que hayamos seleccionado el tipo de recomendador nos aparecen los mapas y escenas donde se utiliza este. A continuación tenemos que seleccionar el usuario que queremos evaluar y escoger una lista de ítems. Al final pulsamos el botón Load Graphics y se muestra un gráfico con el ítem, la valoración que ha dato el usuario, el valor estimado en este momento y el error. Si no disponemos de datos de algún ítem este no saldrá en el gráfico. Si no disponemos de ningún dato se nos indicara con un mensaje.

\chapter{Análisis}

\section{Análisis de requisitos}

\subsection{Requisitos no funcionales}

\begin{table}[H]
	\begin{center}
		\begin{tabular}{|p{1.5cm}| p{10.5cm}|}
			\hline
			Código & Descripción \\
			\hline
			RNF-1  & La aplicación tratara de un simulador. Dicha simulación se realizara sobre mapas online\\ \hline
			RNF-2  & La navegación por los menús de la aplicación se realizara mediante una interfaz gráfica\\ \hline
			RNF-3  & Los textos por defecto de la aplicación serán en inglés\\ \hline
			RNF-4  & La autenfiticación de usuarios se realizará mediante web tokens\\ \hline
			RNF-5  & La interfaz gráfica debe ser responsive desarrollada con bootstrap\\ \hline
			RNF-6  & El back-end debe ser desarrollado con node.js, sockets.io y express \\ \hline
			RNF-7  & El front-end debe ser desarrollado con Angular.js\\ \hline
			RNF-8  & La aplicación permitirá la integración de recomendadores de tipo PUSH y PULL\\ \hline
		\end{tabular}
		\caption{Requisitos no funcionales}
		\label{tabla:requisitosNoFuncionales2}
	\end{center}
\end{table}

\newpage

\subsection{Requisitos funcionales}


\begin{longtable}[H]{|c|p{10cm}|}
	% aquí añadimos el encabezado de la primera hoja.
	\hline
	Código & Descripción \\
	\hline \hline
	\endfirsthead
	
	% aquí añadimos el encabezado del resto de hojas.
	\hline
	Código & Descripción \\
	\hline \hline
	\endhead
	
	% aquí añadimos el fondo de todas las hojas, excepto de la última.
	\multicolumn{2}{c}{}
	\endfoot
	
	% aquí añadimos el fondo de la última hoja.
	\endlastfoot
	
	% aquí añadimos el cuerpo de la tabla.
	RF-1  & La aplicación permitirá crear un nuevo usuario\\ \hline
	RF-2  & La aplicación permitirá al usuario buscar mapas por su nombre, tipo, estado, cuidad y fecha de creación\\ \hline
	RF-3  & La aplicación permitirá al usuario crear un nuevo mapa\\ \hline
	RF-4  & La aplicación permitirá al usuario crear una nueva escena asociada a un mapa existente\\ \hline
	RF-5  & La aplicación listara todas las escenas de un mapa\\ \hline
	RF-6  & La aplicación permitirá al usuario editar un mapa existente\\ \hline
	RF-7  & La aplicación permitirá al usuario editar las escenas de un mapa existente\\ \hline
	RF-8  & La aplicación permitirá al usuario crear un nuevo tipo de objeto estático\\ \hline
	RF-9  & La aplicación permitirá al usuario crear un nuevo tipo de objeto dinámico\\ \hline
	RF-10 & La aplicación listará todos los tipos de objetos estáticos creados\\ \hline
	RF-11 & La aplicación listará todos los tipos de objetos dinámicos creados\\ \hline
	RF-12 & La aplicación permitirá al usuario editar los objetos estáticos credos\\ \hline
	RF-13 & La aplicación permitirá al usuario editar los objetos dinámicos creados\\ \hline
	RF-14 & La aplicación permitirá al usuario cambiar el nombre\\ \hline
	RF-15 & La aplicación permitirá al usuario cambiar su contraseña\\ \hline
	RF-16 & La aplicación permitirá al usuario cambiar la imagen asociada a un usuario\\ \hline
	RF-17 & La aplicación permitirá al usuario configurar un nuevo tipo de recomendador\\ \hline
	RF-18 & La aplicación permitirá al usuario editar la configuración de recomendador existente\\ \hline
	RF-18 & La aplicación permitirá al usuario asociar un recomendador existente a una escena\\ \hline
	RF-19 & La aplicación permitirá al usuario definir los límites de una escena\\ \hline
	RF-20 & La aplicación permitirá al usuario asociar un objeto estático a una escena\\ \hline
	RF-21 & La aplicación permitirá al usuario cargar todos los objetos estáticos desde un fichero JSON\\ \hline
	RF-22 & La aplicación permitirá asociar un objeto dinámico y su definir su ruta en una escena\\ \hline
	RF-23 & La aplicación permitirá al usuario cargar todos los objetos dinámicos y sus rutas desde un fichero JSON\\ \hline
	RF-24 & La aplicación listará todos objetos estáticos asociados a una escena \\ \hline
	RF-25 & La aplicación listará todos los objetos dinámicos asociados a una escena \\ \hline
	RF-26 & La aplicación permitirá borrar un objeto estático asociado a una escena\\ \hline
	RF-27 & La aplicación permitirá borrar un objeto dinámico asociado a una escena\\ \hline
	RF-28 & La aplicación permitirá al usuario elegir un si mapa es colaborativo o no\\ \hline
	RF-29 & La aplicación permitirá al usuario ejecuta una simulación sobre la escena de un mapa\\ \hline
	RF-30 & La aplicación permitirá al usuario solicitar recomendaciones mientras se está ejecutando una simulación siempre y cuando el recomendador asociado a la escena es de tipo pull\\ \hline
	RF-31 & El usuario recibirá recomendaciones sin haberlas solicitado siempre y cuando el recomendador asociado a la escena de es tipo push\\ \hline
	RF-32 & El usuario puede arrancar/pausar una simulación\\ \hline
	RF-33 & La aplicación permitirá al usuario generar de forma aleatoria los grafos de movimiento de los vehículos \\ \hline	
	RF-34 & La aplicación permitirá al usuario valorar cualquier item \\ \hline	
	RF-35 & La aplicación permitirá al usuario recibir recomendaciones de items \\ \hline
	\caption{Requisitos funcionales}
	\label{tabla:requisitosFuncionales2}
\end{longtable}


\section{Objetivos de Usabilidad}

La Usabilidad, según el estándar ISO 9241-11, se define como la medida en la que un producto se puede usar por determinados usuarios para conseguir objetivos específicos de efectividad, eficiencia y satisfacción de un contexto de uso específico.  

Así que para poder garantizar la calidad y la satisfacción de los usuarios tenemos que tener en cuenta los objetivos de usabilidad descritos:  

\begin{itemize}
	\item {\bfseries Efectividad}: asegurar que la aplicación desempeñe correctamente todos los objetivos de la aplicación.  	
	\item {\bfseries Eficiencia}: asegurar que cada objetivo de aplicación sea realizado en el menor tiempo posible desempeñando correctamente su tareas .
	\item {\bfseries Utilidad}:para que el sistema pueda hacer todas las tareas que el usuario deba hacer, la aplicación tendrá conexión a Internet ya que establecerá conexión con un servidor en el que se encuentra toda la información de los productos. Siempre y cuando la conexión sea satisfactoria, el usuario podrá realizar las tareas descritas en el apartado de análisis de requisitos funcionales. 
	\item {\bfseries Seguridad}: asegurar que la aplicación evite situaciones de pérdida de información, evitar que se cuelgue y garantizar la confidencialidad de la información ya que la aplicación tiene acceso a Internet.
\end{itemize}	 

\begin{longtable}[H]{|p{3cm}|p{3cm}|p{3cm}|p{3cm}|}
	% aquí añadimos el encabezado de la primera hoja.
	\hline
	Objetivos & Eficacia & Eficiencia  & Satisfacción \\
	\hline \hline
	\endfirsthead
	
	% aquí añadimos el encabezado del resto de hojas.
	\hline
	Objetivos  & Eficacia & Eficiencia  & Satisfacción \\
	\hline \hline
	\endhead
	
	% aquí añadimos el fondo de todas las hojas, excepto de la última.
	\multicolumn{2}{c}{}
	\endfoot
	
	% aquí añadimos el fondo de la última hoja.
	\endlastfoot
	
	% aquí añadimos el cuerpo de la tabla.
			Utilizabilidad global  & Usuarios que terminan la tarea con éxito: 99\% de usuarios  & Tiempo de realización de tareas:8 seg.  & Frecuencia de quejas: 2 - 4 de cada 100 \\ \hline
			
			Satisface las necesidades de los usuarios habituales  & Tareas terminadas con éxito: 95\% de tareas  & Tiempo de realización de tareas: 5 seg. & Evaluación de satisfacción en el uso de las funciones: 9/10\\ \hline
			
			Satisface las necesidades de los usuarios noveles   & Tareas terminadas con éxito en el primer intento: 90\% de tareas  & Tiempo de realización de las tareas: 15 seg. & Tiempo de uso no obligatorio: 10 seg. - 20 seg. \\ \hline
			
			Facilidad de aprendizaje  & Número de funciones aprendidas: 100\% de las funciones  & Número de usos para aprendizaje: 2 - 3 usos  & Evaluación de la facilidad de aprendizaje: 8/10 \\ \hline
			
			Tolerancia a errores  & Errores registrados o corregidos por el sistema: 100\% de errores & Tiempo empleado en corregir errores: 30 seg. & tratamiento de errores: 9/10 \\ \hline
			
			Legibilidad  & Palabras leídas correctamente a distancia normal: 100\% de palabras  & Tiempo necesarios par leer la pantalla: 10 seg. - 15 seg.  & Evaluación de las molestias visuales: 1/10 (menos nota implica menor molestia)  \\ \hline
	
	\caption{Objetivos de usabilidad}
	\label{tabla:objetivosUsabilidad}
\end{longtable}

\newpage

\section{Diagrama de casos de uso}

En esta sección se mostrará el diagrama de casos de usos analizado. El diagrama de casos de uso se ha divido en dos partes porque es demasiado grande para ser visualizado correctamente en una solo imagen.

\begin{figure}[H]
	\centering\includegraphics[scale=0.35]{imagenes/casos-de-uso-1.png}
	\caption{Diagrama de casos de uso parte 1}
	\label{img:casosDeUso1}
\end{figure}

\begin{figure}[H]
	\centering\includegraphics[scale=0.35]{imagenes/casos-de-uso-2.png}
	\caption{Diagrama de casos de uso parte 2}
	\label{img:casosDeUso2}
\end{figure}

\chapter{Implementación}

En este capítulo se mostraran los fundamentos mas importantes sobre la implementación tanto del front-end como del back-end.

\subsection{Implmentación del front-end}

En esta sección se explicaran cuales son las configuraciones más importantes de Angular.js porque este es el más importante del front-end. 

\subsubsection{Organización y carpetas}

La capeta public es la que contiene todo lo relacionado con el front-end del simulado de escenarios. Dento de este encontramos la siguiente estructura de directorios: 

\dirtree{%
	.1 Simulator.
	.2 public.
	.3 images.
	.3 javascript.
	.4 angular.
	.4 bootstrap.
	.4 jquery.
	.4 openlayers.
	.4 socketsIO.
	.3 stylesheets.
	.3 views.
}

A continuación vamos a ver cual es la función de cada uno de estos directorios:
\begin{itemize}
	\item 
	\item 
	\item 
\end{itemize}

\subsubsection{El router de Angular.js}



\subsubsection{Los controladores de Angular.js}


\subsubsection{Los servicios de Angular.js}


\subsubsection{El factory de Angular.js}


\subsection{Implmentación del back-end}



\subsection{¿Como montar un instalador de la aplicación?}



\chapter{Integración con un recomendador externo}

En este capítulo se mostrará como se realiza la integración con un recomendador externo mediante un sistema bidireccional dirigido por eventos y el intercambio de mensajes JSON.

\section{Introducción}

Tal y como hemos mencionado anteriormente el simulador y el recomendador están integrados mediante un sistema bidireccional dirigido por eventos y el intercambio de mensaje en JSON. Actualmente está desarrollado un recomendador de ejemplo basado en usuarios de tipo pull mediante la librería Apache Mahout. 

La razón por la que se han integrado los dos sistemas mediante un sistema bidireccional dirigido por eventos es que existen recomendadores que realizan recomendaciones sin que el usuario las haya solicitado. En protocolos como el HTTP no podemos enviar respuesta al navegador sin que este haya lanzado un petición previamente mientras que en el sistema bidireccional dirigido por evento si que podemos enviar datos al cliente cuando sea necesario y sin que este los haya solicitado. 

El recomendador de ejemplo desarrollado está pensado en lanzar un hilo para cada tipo de recomendador, es decir, un hilo para el servidor de tipo pull y otro para el recomendador de tipo push. Actualmente solo existe una implementación de un recomendador de tipo pull y por lo tanto solo hay un hilo. Este recomendador incorpora un patrón de diseño de tipo Strategy combinado con un patrón de diseño de tipo Factory que nos permiten cambar el tipo de implementación del recomendador de tipo pull en tiempo de ejecución. 

De esta manera podemos cambiar la implementación del recomendador de tipo pull entre peticiones y dar servicio a los distintos recomendadores configurados en las distintas escenas. 

A continuación se mostrará cual es el modelo de eventos para cada tipo de recomendador y cual es el forma de los mensajes intercambiados. 

\section{Recomendadores pull}

En está sección se mostrará la arquitectura del recomendador pull existente y como expandirlo para crear una nueva implementación para este tipo tipo de recomendador. 

\subsection{¿En que consiste el patrón de diseño de tipo Strategy?}

El patrón Strategy es un patrón de comportamiento porque determina cómo se debe realizar el intercambio de mensajes entre diferentes objetos para resolver una tarea. Permite mantener un conjunto de algoritmos de entre los cuales el cliente puede elegir cual le conviene usar. Tiene los siguiente participantes:

\begin{itemize}
	\item Contexto: es el elemento que usa los algoritmos y delega en la jerarquía de estrategias. Configura una estrategia mediante una referencia a la estrategia concreta.  
	\item Estrategia: una interfaz común para todos los algoritmos. Es usada por el contexto para invocar la estrategia concreta.
	\item Estrategia concreta: implementa el algoritmo utilizando la interfaz definida por la estrategia.
\end{itemize}

En nuestro caso el contexto es el objeto Recommender ubicado en el paquete recommender.context, la interfaz estrategia se corresponde la interfaz Strategy ubicada en el paquete recommender.strategy y la estrategia concreta es la clase UserBasedStrategy ubicada el paquete recommender.strategy. 

En la implementación actual se ha usado en patrón de diseño de tipo Factory para crear y gestionar las instancias de las estrategias concretas. Cuando llega una solicitud de recomendación por un usuario se crea una instancia de la estrategia concreta y se almacena en una tabla hash. Si la estrategia concreta ya este almacenada en la tabla hash entonces solo se devuelve la instancia de esta.

\subsection{Pasos para crear un nuevo recomendador pull}

\subsubsection{Paso 1: Crear una estrategia concreta}

Tal y como hemos mencionado anteriormente la estrategia concreta implementa el algoritmo de recomendación concreto. Para implementar un nuevo tipo de algoritmo de recomendación tenemos que crear una clase en el paquete recommender.strategy que implemente la interfaz Strategy del mismo paquete. 

Esta interfaz tiene dos métodos: recommend y itemForecast. El método recommend es el que se invoca cuando el usuario solicita una recomendación y devuelve una lista de items. El método itemForecast realiza una predicción para un item dado un usuario y devuelve un float que se corresponde a la predicción realizada.

Los parámetros de entrada de ambos métodos son iguales y tienen el mismo significado:

\begin{itemize}
	\item JSONObject data: contiene todos los datos en formato JSON que enviar el simulador.
	\item List$<$Item$>$ itemList: contiene un array list con todos los items a recomendar.
	\item List$<$Ratings$>$ ratingList: contiene un array list con las valoraciones de los usuarios para cada item.
	\item RecommenderConfig recommenderConfig: contiene los parámetros de configuración del recomendador como numero de items a recomendar etc.
\end{itemize}

Todos los datos vienen listos para ser usados por el algoritmo de recomendación que estamos implementando.

\newpage

\subsubsection{Paso 2: expandir el enumerado StrategyType}

StrategyType es un tipo enumerado que contiene todas las estrategias concretas implementadas. Es utilizado por patrón Factory para determinar que tipo de estrategia concreta crear. 

\begin{lstlisting}[language=xml, frame=single]
package recommender.models;

public enum StrategyType {
	UserBasedStrategy("User based recommender");
	
	private String type;
	
	StrategyType(String type) {
		this.type = type;  
	}
	
	public String getType() {
		return this.type;
	}

	public static StrategyType fromString(String type) {
		if (type != null) {
			for (StrategyType b : StrategyType.values()) {
				if (type.equalsIgnoreCase(b.type)) {
					return b;
				}
			}
		}
	
		return null;
	}
}
\end{lstlisting}

En la implementación de la figura anterior podemos ver el siguiente enumerado UserBasedStrategy($"$User based recommender$"$) que se corresponde a la implementación actual del algoritmo de recomendaciones basados en usuarios. La cadena entre comillas es el nombre que se va mostrar en el simulador cuando configuramos un nuevo recomendador.

\subsubsection{Paso 3: expandir el factory StrategyFactory}

Una vez que hayamos expandido StrategyType vamos a expandir el factory StrategyFactory para que pueda usar este nuevo tipo de enumerado y crear la estrategia concreta.

\begin{lstlisting}[language=java, frame=single]
package recommender.strategy;

import java.util.HashMap;
import java.util.Map;
import recommender.models.StrategyType;

public class StrategyFactory {
	static Map<StrategyType, Strategy> map = new HashMap<StrategyType, Strategy>();
	
	public static Strategy createStrategy(StrategyType type){
		Strategy strategy = map.get(type);
		
		if(strategy==null){
			switch(type){
			case UserBasedStrategy: strategy = new UserBasedStrategy(); break;
			default: throw new IllegalArgumentException("The recommender type " + type + " is not recognized.");
			}
			
			map.put(type, strategy);
		}
		
		return strategy;
	}
}
\end{lstlisting}

Para expandirlo vamos al método createStrategy y expandimos el switch para que use el nuevo enumerado que hemos creado en el paso 2 y en función a ellos cree una nueva instancia de la estrategia concreta que hemos creado en el paso 1. 

\subsection{Eventos y mensajes}

En el diagrama siguiente podemos ver cuales son todos los eventos en el recomendador de tipo pull:

\begin{figure}[H]
	\centering\includegraphics[scale=0.5]{imagenes/diagrama-de-eventos.png}
	\caption{Diagrama de eventos del recomendador de tipo pull}
	\label{img:diagramaEventosPull}
\end{figure}

Los eventos importantes del recomendador son:

\begin{itemize}
	\item recommend: es el evento que invoca el simulador para solicitar una recomendación.
	\item getStrategies: es el evento que se invoca para recuperar la lista de estrategias concretas implementadas. No tiene parámetros de entrada.
	\item get value forecast: es el evento que se invoca para solicitar una predicción para un item
\end{itemize}

Los eventos importantes del simulador son:
\begin{itemize}
	\item recommended items: es el evento que invoca el recomendador para enviar la lista de items recomendados.
	\item strategies result: es el evento que invoca el recomendador para enviar la lista de estrategias concretas implementadas.
	\item set value forecast: es el evento que invoca el recomendador para enviar la predicción para un item.
\end{itemize}

A continuación podemos ver los parámetros de entrada de los eventos importantes para una integración con un recomendador externo:

\begin{table}[H]
	\centering
	\label{my-label}
	\begin{tabular}{|l|l|p{7cm}|}
		\hline
		\multicolumn{1}{|c|}{\textbf{Nombre}} & \multicolumn{1}{c|}{\textbf{Tipo}} & \multicolumn{1}{c|}{\textbf{Descripción}}                                                                                   \\ \hline
		userId                                & String                             & token del usuario                                                                                                           \\ \hline
		strategyType                          & String                             & tipo de estrategia concreta que queremos usar para la recomendación. Contiene uno de los valores del enumerado StrategyType \\ \hline
		mapId                                 & String                             & identificador del mapa                                                                                                      \\ \hline
		sceneId                               & String                             & identificador de la escena                                                                                                  \\ \hline
		rating                                & number                             & Valoración del usuario sobre item con identificador itemId                                                                  \\ \hline
		itemId                                & String                             & identificador del item                                                                                                      \\ \hline
		recommenderId                         & String                             & identificador de la configuración del recomendador                                                                          \\ \hline
	\end{tabular}
	\caption{Evento get value forecast del recomendador}
\end{table}

\begin{table}[H]
	\centering
	\label{my-label}
	\begin{tabular}{|l|l|l|}
		\hline
		\multicolumn{1}{|c|}{\textbf{Nombre}} & \multicolumn{1}{c|}{\textbf{Tipo}} & \multicolumn{1}{c|}{\textbf{Descripción}}                    \\ \hline
		itemList                              & Array                              & lista con items que recomendamos al usuario                  \\ \hline
		userList                              & Array                              & lista de tokens de usuarios a los que recomendamos los items \\ \hline
	\end{tabular}
	\caption{Evento recommended items del simulador}
\end{table}

\begin{table}[H]
	\centering
	\label{my-label}
	\begin{tabular}{|l|l|p{6cm}|}
		\hline
		\multicolumn{1}{|c|}{\textbf{Nombre}} & \multicolumn{1}{c|}{\textbf{Tipo}} & \multicolumn{1}{c|}{\textbf{Descripción}}                                                                                                                                                                                                                                                                                                               \\ \hline
		strategyType                          & String                             & tipo de estrategia concreta que queremos usar para la recomendación. Contiene uno de los valores del enumerado StrategyType                                                                                                                                                                                                                             \\ \hline
		mapId                                 & String                             & identificador del map en el que está conectado el usuario                                                                                                                                                                                                                                                                                               \\ \hline
		sceneId                               & String                             & identificador de la escena en la que está conectado el usuario                                                                                                                                                                                                                                                                                          \\ \hline
		token                                 & String                             & token del usuario conectado                                                                                                                                                                                                                                                                                                                             \\ \hline
		recommender                           & String                             & identificador de la configuración del recomendador que estamos usando                                                                                                                                                                                                                                                                                   \\ \hline
		itemTypesToRecommend                  & Array                              & contiene una lista con los identificadores de los tipos de items sobre los que el usuario desea recibir recomendaciones                                                                                                                                                                                                                                 \\ \hline
		resultTypeShow                        & String                             & el tipo de resultado que quiere ver el usuario. Puede contener los siguiente valores: \newline - nonPersonalizedRecommendation: para mostrar recomendaciones no personalizadas en caso que de que el recomendador no tiene items que recomendar \newline - emptyList: para mostrar una lista vacia cuando el recomendador no tiene items que recomendar \\ \hline
		location                              & Object                             & localización actual del usuario. Es un objeto que contiene los siguientes atributos: \newline- longitude: longitud actual \newline - latitude: latitud actual                                                                                                                                                                                           \\ \hline
	\end{tabular}
	\caption{Evento recommend del recomendador}
\end{table}

\newpage

\section{Recomendadores push}

Para crear un recomendador de tipo push tenemos que editar el objeto Server y lanzar un nuevo hilo que se corresponde al recomendador de tipo push y pasarle en el constructor los parámetros configs y socket igual que el PullServer en el siguiente ejemplo: 

\begin{lstlisting}[language=java, frame=single]
public Server() throws URISyntaxException, InterruptedException, IOException{
	try {
		Configurations configs = readBaseConfig();
		String url = configs.getHost()+":"+configs.getPort();
		
		socket = IO.socket(url);
		socket.connect();
		
		//Inicializamos un servidor de tipo pull
		PullServer pullServer = new PullServer(configs, socket);
		pullServer.run();
	} catch (URISyntaxException e) {
		System.out.println(e);
	}
}
\end{lstlisting}

La variable config contiene datos a cerca de la dirección de la base de datos y la dirección del simulador. La variable socket es el sistema bidireccional dirigido por eventos. 

En cuanto a la arquitectura tenemos total liberar para el diseño del recomendador push. Lo único es que tenemos que hacer es crear el evento get strategies push. Este evento no tiene parámetros de entrada y devuelve una lista de enumerados con las implementaciones del recomendador push (ver implementación de la clase StrategyType). Para devolver la lista de recomendaciones tenemos que invocar el evento get strategies push result.

Para enviar las recomendaciones tenemos que usar el evento recommended items del simulador indicando la lista de items que estamos recomendado y la lista de usuarios (tokens) a los que realizamos las recomendaciones.


\chapter*{}
\thispagestyle{empty}

\end{document}
