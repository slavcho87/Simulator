\chapter{Conclusiones}

En este capítulo se explican los resultados que se han obtenido de este Trabajo Fin de Grado, los rendimientos del simulador, el trabajo futuro y una valoración personal.

\section{Resultados}

A lo largo de este Trabajo Fin de Grado se ha desarrollado un simulador de escenarios con usuarios móviles para la evaluación de algoritmos de recomendaciones. El simulador puede ser usado de forma cooperativa por varias personas a través de cualquier dispositivo con conexión a Internet y un navegador web. Cuenta con escenarios basados en datos reales obtenidos a través del servicio de mapas de OpenStreetMap y permite integrar de un recomendador externo.

Como se puede comprobar a continuación, se han cumplido todos los objetivos marcados inicialmente en la propuesta del Trabajo Fin de Grado:
\begin{itemize}
	\item Se he desarrollado un simulador de escenarios con usuarios móviles que cuenta con mapas de ciudades con objetos móviles y estáticos.
	\item Los mapas de las ciudades utilizados en las simulaciones están creados a partir de datos reales obtenidos a través del servicio de mapas de OpenStreetMap.
	\item Los usuarios puedan crear, editar, borrar y configurar los mapas y escenarios del simulador.
	\item Se ha desarrollado un sistema bidireccional basada en eventos que permite integrar un recomendador externo y realizar simulaciones en tiempo real de tal forma que los eventos de los usuarios se reflejen en los dispositivos conectados al mismo mapa y escena.
\end{itemize}

\section{Rendimientos del simulador}

El limite teórico máximo de peticiones de las aplicaciones desarrolladas con Node.js es igual al número máximo de sockets que se puedan crear en el servidor, es decir, un servidor con un Sistema Operativo Linux puede crear 65535 sockets. Pero hay que tener en cuenta que una parte esta reservada para el Sistema Operativo. Esto nos deja entre 30.000 - 45.000 sockets que podemos usar. Tenemos que comprobar como se comporta el simulador y por lo tanto hemos realizado unas prueba de estrés tanto con Linux como con Windows.

\subsection{Rendimientos con Linux}

En este apartado se describe los resultados de las pruebas de estrés realizadas con un Sistema Operativo Linux con las siguientes características:

\begin{itemize}
	\item Sistema Operativo: Linux v15.10 64 bits
	\item RAM: 3.3 GB
	\item CPU: Intel i3 1.8 GHz
\end{itemize}

\begin{table}[H]
	\begin{center}
		\begin{tabular}{|c|c|c|c|c|c|}\hline
			Peticiones & CPU & Memoria & Kbytes/sec & Peticiones/seg & Segundos por petición \\ \hline
			50.000 & 24.8 & 134MB & 1593,86 & 508,76 & 0,196555 \\ \hline
			100.000 & 30.2 & 133 MB & 1716,46 & 547,89 & 0,182517 \\ \hline
			500.000 & 45.4 & 138MB & 1199,38 & 382,84 & 0,261204 \\ \hline
		\end{tabular}
		\caption{Pruebas de estrés con 100 usuarios concurrentes}
		\label{tabla:CienteUsuarios}
	\end{center}
\end{table}

\begin{table}[H]
	\begin{center}
		\begin{tabular}{|c|c|c|c|c|c|} 	\hline
			Peticiones & CPU & Memoria & Kbytes/sec & Peticiones/seg & Segundos por petición \\ \hline
			50.000 & 25.9 & 147MB & 1196,85 & 382,04 & 1,308774 \\ \hline
			100.000 & 40.6 & 160MB & 1286,86 & 410,78 & 1,2172 \\ \hline
			500.000 & 44.6 & 162 MB & n/d & n/d & n/d \\ \hline
		\end{tabular}
		\caption{Pruebas de estrés con 500 usuarios concurrentes}
		\label{tabla:QuinientosUsuarios}
	\end{center}
\end{table}

\begin{table}[H]
	\begin{center}
		\begin{tabular}{|c|c|c|c|c|c|} \hline
			Peticiones & CPU & Memoria & Kbytes/sec & Peticiones/seg & Segundos por petición \\ \hline
			50.000 & 24.3 & 160 MB & 1408,94 & 449,74 & 2,223517 \\ \hline
			100.000 & 26.3 & 167MB & n/d & n/d & n/d \\ \hline
			500.000 & n/d & n/d & n/d & n/d & n/d \\ \hline
		\end{tabular}
		\caption{Pruebas de estrés con 1000 usuarios concurrentes}
		\label{tabla:MilUsuarios}
	\end{center}
\end{table}

En el caso de la pruebas con 500 usuarios concurrentes nos da dado el error apr socket recv Connection timed out (110) y se han completado 105.195 peticiones en la prueba con 500.000 peticiones. En el caso de la prueba con 1000 usuarios concurrentes nos ha dado el mismo fallo que en el caso de los 500 usuarios concurrentes y se han completado 54.599 peticiones en la prueba con 100.000 peticiones. 

En el resto de casos las pruebas de carga han sido satisfactorias. Observamos que el peor de los casos se ha obtenido una media de 2,2 segundos por petición en el caso que tenemos 1000 usuarios concurrentes que han generado 50.000 peticiones.

\subsection{Rendimientos con Windows}

En este apartado se describe los resultados de las pruebas de estrés realizadas con un Sistema Operativo Windows con las siguientes características:

\begin{itemize}
	\item Sistema Operativo: Windows 10 Home 64 bits
	\item RAM: 8 GB
	\item CPU: Intel i3 1.8 GHz
\end{itemize}

\begin{lstlisting}[language=bash]
slavcho@ubuntu:~/testing$ ab -g resultados4.csv -n 50000 -c 100 http://192.168.1.102:81/
This is ApacheBench, Version 2.3 <$Revision: 1638069 $>
Copyright 1996 Adam Twiss, Zeus Technology Ltd, http://www.zeustech.net/
Licensed to The Apache Software Foundation, http://www.apache.org/

Benchmarking 192.168.1.102 (be patient)
Completed 5000 requests
Completed 10000 requests
Completed 15000 requests
apr_socket_recv: Connection timed out (110)
Total of 19219 requests completed
slavcho@ubuntu:~/testing$
\end{lstlisting}

\newpage

\begin{lstlisting}[language=bash]
slavcho@ubuntu:~/testing$ ab -g resultados4.csv -n 50000 -c 500 http://192.168.1.102:81/
This is ApacheBench, Version 2.3 <$Revision: 1638069 $>
Copyright 1996 Adam Twiss, Zeus Technology Ltd, http://www.zeustech.net/
Licensed to The Apache Software Foundation, http://www.apache.org/

Benchmarking 192.168.1.102 (be patient)
apr_socket_recv: Connection refused (111)
Total of 558 requests completed
\end{lstlisting}


\begin{lstlisting}[language=bash]
slavcho@ubuntu:~/testing$ ab -g resultados4.csv -n 100000 -c 500 http://192.168.1.102:81/
This is ApacheBench, Version 2.3 <$Revision: 1638069 $>
Copyright 1996 Adam Twiss, Zeus Technology Ltd, http://www.zeustech.net/
Licensed to The Apache Software Foundation, http://www.apache.org/

Benchmarking 192.168.1.102 (be patient)
apr_socket_recv: Connection refused (111)
Total of 524 requests completed
\end{lstlisting}

En las pruebas realizadas observamos que nos ha dado fallo en todas las pruebas. En el caso de la prueba con 100 usuarios concurrentes 50.000 peticiones se ha completado 19.219 de las peticiones. En el caso de la prueba con 500 usuarios concurrentes y 50.000 peticiones se han completado 558 peticiones. En el caso de la prueba con 500 usuarios concurrentes y 100.000 peticiones se han completado 524 peticiones. 

\subsection{Conclusiones de las pruebas de estrés}

Como conclusión podemos determinar que el Sistema Operativo Linux se comporta bastante mejor que el Windows para esta aplicación. Por lo tanto recomiendo la utilización de un servidor Linux para el simulador de escenarios para aprovechar mejor tanto la aplicación como el hardware.

\section{Trabajos futuros}

A continuación se proponen algunas posibles mejoras futuras:

\begin{itemize}
	\item implementar lo necesario para simular eventos sobre el mapa como pueden ser las calles cortadas, ofertas de descuentos en algún local cierto día o días de la semana etc.
	\item implementar lo necesario para exportar los datos del mapa: tanto el grafo de carreteras como la configuración de los objetos estáticos y dinámicos.
	\item implementar lo necesario para importar datos exportados.
	\item implementar lo necesario para que los objetos dinámicos se generen y se muevan por el mapa mediante distintas técnicas de la Inteligencia Artificial.
	\item implementar lo necesario para que el servidor web Node.js pueda ejecutarse sobre multiples hilos aprovechando mejor las CPUs actuales. Actualmente solo se ejecuta sobre un proceso de un solo hilo y para escalar el sistema existe la necesidad de clusterizar el servidor web.
	\item utilizar algún framework de Javascript que nos permita desarrollar pruebas automaticas tanto al lado del cliente como al lado del servidor.
\end{itemize}

\section{Valoración personal}

El trabajo realizado ha sido muy satisfactorio ya que me ha aportado muchos conocimientos relacionados con la rama de Sistemas de Información en la que he estado especializandome y además me ha permitido profundizar en muchas tecnologías como Node.js y Angular.js en las que he estado interesado por el auge que están sufriendo en al actualidad. Además creo los conocimientos adquiridos me dan valor añadido como futuro profesional del sector de las tecnologías de la información.

Aún así durante el tiempo de desarrollo de todo el proyecto existían preguntas que he estado haciendome continuamente. ¿Si las tecnologías y decisiones que he tomado han sido las adecuadas? ¿Es fácil de implementar y sin perder tiempo? ¿Sería fácil de mantener el sistema en el futuro? ¿Las tecnologías elegidas tendrían algún limite que no permita o dificulte la implementación de ciertas funcionalidades en el futuro? 

Como conclusión personal he obtenido que no existe la perfección y de lo complicado que es desarrollar una buena aplicación. Lo que realmente importa es ser constante y comprometido con el trabajo que se está haciendo y aprender a ser autodidacta.