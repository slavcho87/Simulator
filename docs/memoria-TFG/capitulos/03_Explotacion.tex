\chapter{Explotación}

En este capítulo se expone la explotación del simulador de escenarios como método para la evaluación de algoritmos de recomendaciones, se expone el trabajo realizado para permitirlo y el rendimiento obtenido del simulador.

\section{Motivación}

Debido a la gran cantidad de información resulta difícil para los usuarios elegir entre todas las alternativas existentes en los resultados mostrados. Por ello, para facilitar la elección se utilizan los llamados sistemas de recomendación. Estos sistemas resultan de gran interés para las empresas desde punto de vista económico, ya que hacen llegar a sus cliente recomendaciones de productos relevantes para ellos. Ademas, también resultan de interés para los usuarios ya que actúan como filtro y les ofrecen información personalizada que les resulta de interés.

No obstante, el diseño de sistemas de recomendación se enfrenta a problema como el arranque en frió (cold
start problem) o el problema de opiniones artificiales o manipuladas (spam). Esto conlleva que el sistema de recomendación muestra información no relevante para el usuario y este dejará de confiar en él y abandonarlo. 

Por esto surge la necesidad de que los sistemas de recomendaciones puedan evaluarse y calibrarse correctamente. Para que esto pueda llevarse a cabo necesitamos recolectar datos reales de escenarios reales para su posterior análisis. Este análisis consiste en calcular el error cometido por el sistema de recomendación. Una forma de cuantificar el error es con la MAE (Medium Absolute Error) que consiste en restar el valor asignado por el usuario del valor estimado por el sistema de recomendación. 

\section{Definir un escenario realista}




\section{Evaluación de un sistema de recomendación}


\section{Rendimiento}


