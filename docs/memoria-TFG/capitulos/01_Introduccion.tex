\chapter{Introducción}

En este capítulo se mostrará la motivación existente para la realización de este Trabajo Fin de Grado, los objetivos que han sido marcados por el proyecto, las librerías y herramientas utilizadas para su elaboración, el modelo de trabajo seleccionado y también se analizará el contexto tecnológico. Finalmente se mostrará la estructura seguida en este documento.

\section{Motivación del proyecto}

Han sido varias las razones que me llevaron a elegir desarrollar este Trabajo Fin de Grado. La primera y principal ha sido el interés personal en los sistemas de recomendaciones y su amplia aplicación en sistemas comerciales y la web. Por otro lado, realizar un proyecto complejo, partiendo desde cero y sin tener ningún conocimiento particular de este ámbito, suponía un gran reto que deseaba afrontar porque me permitiría ampliar mis conocimientos en campos diversos como Ingeniería del Software, Arquitecturas de Software etc., de las que poseía unos conocimientos limitados. Además, consideré que la experiencia y conocimientos que adquiriría en este proyecto aumentarían mis posibilidades de desarrollar mi carrera profesional en este ámbito.

\section{Objetivos}

El Trabajo Fin de Grado que se describe en este documento tiene los siguientes objetivos:

\begin{itemize}
       \item Desarrollar un simulador de escenarios con usuarios móviles, que cuente con mapas de ciudades con objetos móviles y estáticos.
       \item Desarrollar lo necesario para que se permita que los mapas de ciudades sean reales de tal forma que estos sean obtenidos de un sistema que proporcione mapas.
       \item Desarrollar lo necesario para que el simulador permita la simulación y recomendación de items teniendo en cuenta las ubicaciones geográficas de tanto de los usuarios como de los items a recomendar.
       \item Desarrollar lo necesario para que los usuarios puedan crear, editar, borrar y configurar los mapas y escenarios del simulador.
       \item Desarrollar lo necesario para que todas las configuraciones de mapas y escenarios sean parametrizables desde la interfaz de usuario.
       \item Desarrollar lo necesario para que la simulación de escenas funcione en tiempo real de tal forma que los eventos de una escena se reflejen en los dispositivos de los usuarios conectados al mismo mapa y escena.
       \item Desarrollar una interfaz que permita la integración con un recomendador externo de tal forma que exista una comunicación bidireccional basada en eventos entre el simulador y el recomendador.
\end{itemize}

Además de los objetivos marcados por la propuesta del Trabajo Fin de Grado, también se han tenido en cuenta como objetivos lograr que el simulador utilice los recursos hardware mínimos, permitir que este sea fácilmente escalable y que pueda desplegarse en un entorno distribuido. De esta forma logramos ahorrar costes de infraestructura y futuros desarrollos.

\section{Contexto Tecnológico}

Habitualmente los sistemas de recomendación se evalúan con conjuntos de datos estáticos clásicos. El inconveniente que esto conlleva es que solo podemos realizar un análisis sobre el comportamiento de los usuarios en el pasado perdiendo la capacidad de poder calibrar correctamente los algoritmos de recomendaciones para su correcta evaluación. Con el fin de evitar este inconveniente surge la necesidad de evaluar los algoritmos de recomendaciones con conjuntos de datos dinámicos.

Una de las formas de recolectar y proveer conjuntos de datos dinámicos es mediante el crowdsourcing. El crowdsourcing, también conocido como colaboración abierta distribuida, consiste en delegar tareas a un grupo de personas o comunidad a través de una convocatoria abierta. 

Así que a medida que el grupo de personas o comunidad vayan realizando las tareas asignadas el sistema va recolectado y suministrando datos sobre las acciones que realizan estos para llevar a cabo sus tareas. Esto nos da la oportunidad para suministrar datos a los algoritmos de recomendaciones en tiempo real, es decir, suministrar datos a medida que el grupo de personas vayan cumpliendo con sus tareas.

Cabe destacar que en este Trabajo Fin de Grado no se utiliza crowdsourcing sino se centra solo en la parte de recolección de datos en tiempo real basándose en las acciones o eventos que realizan los usuarios en el entorno de una ciudad real. De esta forma obtendremos conjuntos de datos dinámicos para la correcta calibración y evaluación de los algoritmos de recomendaciones. 

\section{Herramientas utilizadas}

En esta sección se listan las tecnologías, liberarías externas y herramientas utilizadas para el desarrollo del proyecto acompañada de una breve descripción.

\noindent{\textbf{Librerías usadas}}\\

Para el desarrollo del proyecto se ha hecho uso de diversas librerías externas que han permitido la implementación en un tiempo razonable de ciertas funciones necesarias que no formaban parte de los objetivos del proyecto:

\begin{itemize}
       \item {\bfseries Openlayers}: framework de OpenStreetMap que nos permite el uso libre de mapas.
       \item {\bfseries Node.js v0.12.4}: entorno Javascript del lado del servidor basado en el motor V8 de Google. 
       \item {\bfseries Express v4.12.4}: framework de Node.js destinado a la creación de APIs Rest
       \item {\bfseries Angular.js}: framework javascript que facilita la creación de aplicaciones en una sola página. 
       \item {\bfseries Mongoose v4.1.2}: framework de Node.js destinado al modelado de objetos para MongoDB.
       \item {\bfseries jwt.io v5.0.4}: framework destinado a la creación y distribución de web tokens. 
       \item {\bfseries socket.io v1.4.4}: framework de Node.js destinado a la creación de comunicaciones bidirecionales basadas en eventos
       \item {\bfseries Apache mahout v0.11.1}: framework de Java destinado al aprendizaje automático.
       \item {\bfseries socket.io-client v0.1.0}: cliente Java para socket.io desarrollado por Naoyuki Kanezawa
\end{itemize}
    
\noindent{\textbf{Herramienta de desarrollo}}\\

Durante el desarrollo de este Trabajo Fin de Grado se han utilizado las siguientes herramientas:

\begin{itemize}
       \item {\bfseries Eclipse Java EE IDE}: editor de código Java versión Mars 4.5.1
       \item {\bfseries Maven v3.2.5}: gestor de paquetes para desarrollos Java
       \item {\bfseries Brackets.io v1.6.0}: editor de código para desarrollos web
       \item {\bfseries Git v1.9.4}: sistema de control de versiones
       \item {\bfseries GitHub}: repositorio de código
       \item {\bfseries Cmder}: emulador cmd de Windows 
       \item {\bfseries Sublime text 2}: editor de texto avanzado
\end{itemize}

\noindent{\textbf{Herramienta de documentación}}\\

Se han usado las siguientes herramientas para la elaboración de la documentación del proyecto:

\begin{itemize}
       \item {\bfseries Latex}: lenguaje usado para la elaboración de este documento
       \item {\bfseries GanttProject}: editor de diagramas de Gantt
       \item {\bfseries Gliffy Diagrams}: editor de múltiples tipos de diagramas (diagramas de flujos, UML, etc.)
       \item {\bfseries Apache Benchmark}: herramienta para hacer pruebas de carga de sitios web
       \item {\bfseries http://newrelic.com/}: sitio web para motorización de servidores
\end{itemize}


\section{Modelo de proceso seleccionado}

El modelo de trabajo seleccionado está basado en el modelo de espiral. Las actividades de este modelo forman una espiral de tal forma que cada iteración representa un conjunto de actividades. Se ha elegido este modelo de trabajo porque nos permitiría integrar el desarrollo con el mantenimiento y evaluar en cada iteración si dichos requisitos siguen encajando de lo que se esperaba de la aplicación para conseguir los objetivos propuestos. De esta forma se reduce el riesgo del proyecto y se incorporan objetivos de calidad.

\section{Estructura de la memoria}

El contenido de la memoria está distribuido de la siguiente forma:

\begin{itemize}
       \item En el capítulo 2 se expone el trabajo desarrollado para la elaboración de este simulador
       \item En el capítulo 3 se expone la posible explotación del simulador de escenarios. También se muestra el rendimiento obtenido por el simulador.
       \item En el capítulo 4 se expone la gestión del proyecto y las distintas etapas por las que ha pasado este proyecto
       \item En el capítulo 5 se muestran las conclusiones del proyecto y el posible trabajo futuro de cara a mejorar el simulador 
\end{itemize}

Respecto al contenido de los anexos:

\begin{itemize}
	\item En el anexo A contiene un manual de usuario.
	\item En el anexo B se muestra todo lo referente al análisis realizado.
	\item En el anexo C se muestra todo lo referente a la implementación realizada.
	\item En el anexo D se muestra todo lo referente a la integración de un recomendador externo.
	\item En el anexo E contiene un articulo a presentar en las XXI Jornadas de Ingeniería del Software y Bases de Datos.
\end{itemize}