\chapter{Conclusiones}

En este capítulo se explican los resultados que se han obtenido de este Trabajo Fin de Grado, los rendimientos del simulador, el trabajo futuro y una valoración personal.

\section{Resultados}

A lo largo de este Trabajo Fin de Grado se ha desarrollado un simulador de escenarios con usuarios móviles para la evaluación de algoritmos de recomendaciones. El simulador puede ser usado de forma cooperativa por varias personas a través de cualquier dispositivo con conexión a Internet y un navegador web. Cuenta con escenarios basados en datos reales obtenidos a través del servicio de mapas de OpenStreetMap y permite integrar de un recomendador externo.

Como se puede comprobar a continuación, se han cumplido todos los objetivos marcados inicialmente en la propuesta del Trabajo Fin de Grado:
\begin{itemize}
	\item Se ha desarrollado un simulador de escenarios con usuarios móviles que cuenta con mapas de ciudades con objetos móviles y estáticos.
	\item Los mapas de las ciudades utilizados en las simulaciones están creados a partir de datos reales obtenidos a través del servicio de mapas de OpenStreetMap.
	\item Los usuarios puedan crear, editar, borrar y configurar los mapas y escenarios del simulador.
	\item Se ha desarrollado un sistema bidireccional basada en eventos que permite integrar un recomendador externo y realizar simulaciones en tiempo real de tal forma que los eventos de los usuarios se reflejen en los dispositivos conectados al mismo mapa y escena.
\end{itemize}

\section{Trabajos futuros}

A continuación se proponen algunas posibles mejoras futuras:

\begin{itemize}
	\item implementar lo necesario para simular eventos sobre el mapa como pueden ser las calles cortadas, ofertas de descuentos en algún local cierto día o días de la semana etc.
	\item implementar lo necesario para exportar los datos del mapa: tanto el grafo de carreteras como la configuración de los objetos estáticos y dinámicos.
	\item implementar lo necesario para importar datos exportados.
	\item implementar lo necesario para que el servidor web Node.js pueda ejecutarse sobre multiples hilos aprovechando mejor las CPUs actuales. Actualmente solo se ejecuta sobre un proceso de un solo hilo y para escalar el sistema existe la necesidad de clusterizar el servidor web.
	\item utilizar algún framework de Javascript que nos permita desarrollar pruebas automaticas tanto al lado del cliente como al lado del servidor.
\end{itemize}

\section{Valoración personal}

El trabajo realizado ha sido muy satisfactorio ya que me ha aportado muchos conocimientos relacionados con la rama de Sistemas de Información en la que he estado especializandome y además me ha permitido profundizar en muchas tecnologías como Node.js y Angular.js en las que he estado interesado por el auge que están sufriendo en al actualidad. Además creo los conocimientos adquiridos me dan valor añadido como futuro profesional del sector de las tecnologías de la información.

Aún así durante el tiempo de desarrollo de todo el proyecto existían preguntas que he estado haciendome continuamente. ¿Si las tecnologías y decisiones que he tomado han sido las adecuadas? ¿Es fácil de implementar y sin perder tiempo? ¿Sería fácil de mantener el sistema en el futuro? ¿Las tecnologías elegidas tendrían algún limite que no permita o dificulte la implementación de ciertas funcionalidades en el futuro? 

Como conclusión personal he obtenido que no existe la perfección y de lo complicado que es desarrollar una buena aplicación. Lo que realmente importa es ser constante y comprometido con el trabajo que se está haciendo y aprender a ser autodidacta.