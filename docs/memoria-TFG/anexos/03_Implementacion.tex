\chapter{Implementación}

En este capítulo se mostraran los fundamentos mas importantes sobre la implementación tanto del front-end como del back-end.

\subsection{Implmentación del front-end}

En esta sección se explicaran cuales son las configuraciones más importantes de Angular.js porque este es el más importante del front-end. 

\subsubsection{Organización y carpetas}

La capeta public es la que contiene todo lo relacionado con el front-end del simulado de escenarios. Dento de este encontramos la siguiente estructura de directorios: 

\dirtree{%
	.1 Simulator.
	.2 public.
	.3 images.
	.3 javascript.
	.4 angular.
	.4 bootstrap.
	.4 jquery.
	.4 openlayers.
	.4 socketsIO.
	.3 stylesheets.
	.3 views.
}

A continuación vamos a ver cual es la función de cada uno de estos directorios:
\begin{itemize}
	\item 
	\item 
	\item 
\end{itemize}

\subsubsection{El router de Angular.js}



\subsubsection{Los controladores de Angular.js}


\subsubsection{Los servicios de Angular.js}


\subsubsection{El factory de Angular.js}


\subsection{Implmentación del back-end}



\subsection{¿Como montar un instalador de la aplicación?}


