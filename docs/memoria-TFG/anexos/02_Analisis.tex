\chapter{Análisis}

\section{Análisis de requisitos}

\subsection{Requisitos no funcionales}

\begin{table}[H]
	\begin{center}
		\begin{tabular}{|p{1.5cm}| p{10.5cm}|}
			\hline
			Código & Descripción \\
			\hline
			RNF-1  & La aplicación tratara de un simulador. Dicha simulación se realizara sobre mapas online\\ \hline
			RNF-2  & La navegación por los menús de la aplicación se realizara mediante una interfaz gráfica\\ \hline
			RNF-3  & Los textos por defecto de la aplicación serán en inglés\\ \hline
			RNF-4  & La autenfiticación de usuarios se realizará mediante web tokens\\ \hline
			RNF-5  & La interfaz gráfica debe ser responsive desarrollada con bootstrap\\ \hline
			RNF-6  & El back-end debe ser desarrollado con node.js, sockets.io y express \\ \hline
			RNF-7  & El front-end debe ser desarrollado con Angular.js\\ \hline
		\end{tabular}
		\caption{Requisitos no funcionales}
		\label{tabla:requisitosNoFuncionales2}
	\end{center}
\end{table}

\newpage

\subsection{Requisitos funcionales}


\begin{longtable}[H]{|c|p{10cm}|}
	% aquí añadimos el encabezado de la primera hoja.
	\hline
	Código & Descripción \\
	\hline \hline
	\endfirsthead
	
	% aquí añadimos el encabezado del resto de hojas.
	\hline
	Código & Descripción \\
	\hline \hline
	\endhead
	
	% aquí añadimos el fondo de todas las hojas, excepto de la última.
	\multicolumn{2}{c}{}
	\endfoot
	
	% aquí añadimos el fondo de la última hoja.
	\endlastfoot
	
	% aquí añadimos el cuerpo de la tabla.
	RF-1  & La aplicación permitirá crear un nuevo usuario\\ \hline
	RF-2  & La aplicación permitirá al usuario buscar mapas por su nombre, tipo, estado, cuidad y fecha de creación\\ \hline
	RF-3  & La aplicación permitirá al usuario crear un nuevo mapa\\ \hline
	RF-4  & La aplicación permitirá al usuario crear una nueva escena asociada a un mapa existente\\ \hline
	RF-5  & La aplicación listara todas las escenas de un mapa\\ \hline
	RF-6  & La aplicación permitirá al usuario editar un mapa existente\\ \hline
	RF-7  & La aplicación permitirá al usuario editar las escenas de un mapa existente\\ \hline
	RF-8  & La aplicación permitirá al usuario crear un nuevo tipo de objeto estático\\ \hline
	RF-9  & La aplicación permitirá al usuario crear un nuevo tipo de objeto dinámico\\ \hline
	RF-10 & La aplicación listará todos los tipos de objetos estáticos creados\\ \hline
	RF-11 & La aplicación listará todos los tipos de objetos dinámicos creados\\ \hline
	RF-12 & La aplicación permitirá al usuario editar los objetos estáticos credos\\ \hline
	RF-13 & La aplicación permitirá al usuario editar los objetos dinámicos creados\\ \hline
	RF-14 & La aplicación permitirá al usuario cambiar el nombre\\ \hline
	RF-15 & La aplicación permitirá al usuario cambiar su contraseña\\ \hline
	RF-16 & La aplicación permitirá al usuario cambiar la imagen asociada a un usuario\\ \hline
	RF-17 & La aplicación permitirá al usuario configurar un nuevo tipo de recomendador\\ \hline
	RF-18 & La aplicación permitirá al usuario editar la configuración de recomendador existente\\ \hline
	RF-18 & La aplicación permitirá al usuario asociar un recomendador existente a una escena\\ \hline
	RF-19 & La aplicación permitirá al usuario definir los limites de una escena\\ \hline
	RF-20 & La aplicación permitirá al usuario asociar un objeto estático a una escena\\ \hline
	RF-21 & La aplicación permitirá al usuario cargar todos los objetos estáticos desde un fichero JSON\\ \hline
	RF-22 & La aplicación permitirá asociar un objeto dinámico y su definir su ruta en una escena\\ \hline
	RF-23 & La aplicación permitirá al usuario cargar todos los objetos dinámicos y sus rutas desde un fichero JSON\\ \hline
	RF-24 & La aplicación listará todos objetos estáticos asociados a una escena \\ \hline
	RF-25 & La aplicación listará todos los objetos dinámicos asociados a una escena \\ \hline
	RF-26 & La aplicación permitirá borrar un objeto estático asociado a una escena\\ \hline
	RF-27 & La aplicación permitirá borrar un objeto dinámico asociado a una escena\\ \hline
	RF-28 & La aplicación permitirá al usuario elegir un si mapa es colaborativo o no\\ \hline
	RF-29 & La aplicación permitirá al usuario ejecuta una simulación sobre la escena de un mapa\\ \hline
	RF-30 & La aplicación permitirá al usuario solicitar recomendaciones mientras se está ejecutando una simulación siempre y cuando el recomendador asociado a la escena es de tipo pull\\ \hline
	RF-31 & El usuario recibirá recomendaciones sin haberlas solicitado siempre y cuando el recomendador asociado a la escena de es tipo push\\ \hline
	RF-32 & El usuario puede arrancar/pausar una simulación\\ \hline
	RF-33 & La aplicación permitirá al usuario generar de forma aleatoria los grafos de movimiento de los vehículos \\ \hline	
	\caption{Requisitos funcionales}
	\label{tabla:requisitosFuncionales2}
\end{longtable}


\section{Objetivos de Usabilidad}

La Usabilidad, según el estándar ISO 9241-11, se define como la medida en la que un producto se puede usar por determinados usuarios para conseguir objetivos específicos de efectividad, eficiencia y satisfacción de un contexto de uso específico.  

Así que para poder garantizar la calidad y la satisfacción de los usuarios tenemos que tener en cuenta los objetivos de usabilidad descritos:  

\begin{itemize}
	\item {\bfseries Efectividad}: asegurar que la aplicación desempeñe correctamente todos los objetivos de la aplicación.  	
	\item {\bfseries Eficiencia}: asegurar que cada objetivo de aplicación sea realizado en el menor tiempo posible desempeñando correctamente su tareas .
	\item {\bfseries Utilidad}:para que el sistema pueda hacer todas las tareas que el usuario deba hacer, la aplicación tendrá conexión a Internet ya que establecerá conexión con un servidor en el que se encuentra toda la información de los productos. Siempre y cuando la conexión sea satisfactoria, el usuario podrá realizar las tareas descritas en el apartado de análisis de requisitos funcionales. 
	\item {\bfseries Seguridad}: asegurar que aplicación evite situaciones de pérdida de información, evitar que se cuelgue y garantizar la confidencialidad de la información ya que la aplicación tiene acceso a Internet.
\end{itemize}	 

\begin{longtable}[H]{|p{3cm}|p{3cm}|p{3cm}|p{3cm}|}
	% aquí añadimos el encabezado de la primera hoja.
	\hline
	Objetivos & Eficacia & Eficiencia  & Satisfacción \\
	\hline \hline
	\endfirsthead
	
	% aquí añadimos el encabezado del resto de hojas.
	\hline
	Objetivos  & Eficacia & Eficiencia  & Satisfacción \\
	\hline \hline
	\endhead
	
	% aquí añadimos el fondo de todas las hojas, excepto de la última.
	\multicolumn{2}{c}{}
	\endfoot
	
	% aquí añadimos el fondo de la última hoja.
	\endlastfoot
	
	% aquí añadimos el cuerpo de la tabla.
			Utilizabilidad global  & Usuarios que terminan la tarea con éxito: 99\% de usuarios  & Tiempo de realización de tareas:8 seg.  & Frecuencia de quejas: 2 - 4 de cada 100 \\ \hline
			
			Satisface las necesidades de los usuarios habituales  & Tareas terminadas con éxito: 95\% de tareas  & Tiempo de realización de tareas: 5 seg. & Evaluación de satisfacción en el uso de las funciones: 9/10\\ \hline
			
			Satisface las necesidades de los usuarios noveles   & Tareas terminadas con éxito en el primer intento: 90\% de tareas  & Tiempo de realización de las tareas: 15 seg. & Tiempo de uso no obligatorio: 10 seg. - 20 seg. \\ \hline
			
			Facilidad de aprendizaje  & Número de funciones aprendidas: 100\% de las funciones  & Número de usos para aprendizaje: 2 - 3 usos  & Evaluación de la facilidad de aprendizaje: 8/10 \\ \hline
			
			Tolerancia a errores  & Errores registrados o corregidos por el sistema: 100\% de errores & Tiempo empleado en corregir errores: 30 seg. & tratamiento de errores: 9/10 \\ \hline
			
			Legibilidad  & Palabras leídas correctamente a distancia normal: 100\% de palabras  & Tiempo necesarios par leer la pantalla: 10 seg. - 15 seg.  & Evaluación de las molestias visuales: 1/10 (menos nota implica menor molestia)  \\ \hline
	
	\caption{Objetivos de usabilidad}
	\label{tabla:objetivosUsabilidad}
\end{longtable}

\newpage

\section{Diagrama de casos de uso}

En esta sección se mostrarán los diagramas de casos de usos analizados. El análisis se ha dividido entre, por un lado, la navegación por los menús y por otro lado la simulación en sí. 



\section{Diagrama de casos de uso detallados}


