\chapter*{}

\cleardoublepage
\thispagestyle{empty}

\begin{center}
{\large\bfseries \myTitle}\\
\end{center}

\vspace{0.7cm}
\noindent{\textbf{Resumen}}\\

Los sistemas de recomendación proporcionan sugerencias acerca de elementos que pueden resultar de interés para el usuario (hoteles, restaurantes, libros, películas, etc.). En este TFG se pretende desarrollar un simulador de escenarios con usuarios móviles (mapas de ciudades con objetos móviles y estáticos) que permita la evaluación de algoritmos de recomendación. 

\cleardoublepage


\thispagestyle{empty}


\chapter*{Agradecimientos}
\thispagestyle{empty}

       \vspace{1cm}


Me gustaría agradecer este Trabajo Fin de Grado a todas las personas que lo han echo posible con su apoyo y dedicación.

\vspace{1cm}
En su primer lugar a mi director Sergio Ilarri por la oportunidad que me ha dado para realizar este proyecto, su paciencia y ayuda, sin la cual este proyecto no hubiera sido posible. A mis compañeros y amigos de clase, con los que he compartido estos años de carrera, por hacer que los momentos de estudio y prácticas fuesen agradables y amenos. A mi familia y amigos más cercanos, por su paciencia y por motivarme para seguir adelante en los momentos más complicados.

\vspace{1cm}
Y por supuesto a la Universidad de Zaragoza y a todos aquellos profesores de lo que he aprendido tanto a los largo de estos años.

\cleardoublepage

\tableofcontents % indice de contenidos

\cleardoublepage
\addcontentsline{toc}{chapter}{Lista de figuras} % para que aparezca en el indice de contenidos
\listoffigures % indice de figuras

\cleardoublepage
\addcontentsline{toc}{chapter}{Lista de tablas} % para que aparezca en el indice de contenidos
\listoftables % indice de tablas
