\chapter{Introducción}
\thispagestyle{empty}

En este capítulo se mostrará la motivación existente para la realización de este Trabajo Fin de Grado, los objetivos que han sido marcados por el proyecto, las librerías y herramientas utilizadas para su elaboración, el modelo de trabajo seleccionado y también se analizará el trabajo relacionado. Finalmente se mostrará la estructura seguida en este documento.

\section{Motivación del proyecto}
\thispagestyle{empty}

       \vspace{1cm}

Han sido varias las razones que me llevaron a elegir desarrollar este Trabajo Fin de Grado. La primera y principal ha sido el interés personal en los sistemas de recomendaciones y su amplia aplicación en sistemas comerciales. Por otro lado, realizar un proyecto complejo, partiendo desde cero y sin tener ningún conocimiento particular de este ámbito, suponía un gran reto que deseaba afrontar porque me permitiría ampliar mis conocimientos en campos diversos como Ingeniería del Software, Arquitecturas de Software, Inteligecia Artificial etc., de las que poseía unos conocimientos limitados. Además, consideré que la experiencia y conocimientos que adquiriría en este proyecto aumentarian mis posibilidades de desarrollar mi carrera profesional en este ámbito.

\section{Objetivos}
\thispagestyle{empty}

       \vspace{1cm}

El Trabajo Fin de Grado que se describe en este documento tiene los siguientes objetivos:

\begin{itemize}
	\item objetivo 1
	\item objetivo 2
\end{itemize}

%Además de estos objetivos marcados por la propuesta del Trabajo Fin de Grado, también se ha tenido en cuenta como objetivo lograr el simulador ...


\section{Herramientas utilizadas}
\thispagestyle{empty}

       \vspace{1cm}

En esta sección se listan las tecnologias, libererías externas y herramientas utilizadas para el desarrollo del proyecto acompañada de una breve descripción del porqué de su uso.

\section{Modelo de proceso seleccionado}
\thispagestyle{empty}

       \vspace{1cm}

El modelo de trabajo seleccionado está basado en el modelo de espiral. Las actividades de este modelo forman una espiral de tal forma que cada iteración representa un conjunto de actividades. Se ha elegido este modelo de trabajo porque nos permitiría integrar el desarrollo con el mantenimiento y evaluar en cada iteración si dichos requisitos siguen encajando de lo que se esperaba de la aplicación para conseguir los objetivos propuestos. De esta forma se reduce el riesgo del proyecto y se incorporan objetivos de calidad.


\section{Trabajos relacionados}
\thispagestyle{empty}

       \vspace{1cm}

Poner los trabajos relacionados


\section{Estructura de la memoria}
\thispagestyle{empty}
       \vspace{1cm}

El contenido de la memoría está distribuido de la siguiente forma:

\begin{itemize}
	\item En el capítulo 2 se expone el trabajo desarrollado para la elaboración del simulador
	\item En el capítulo 3 se analiza la posible explotación del simulador como método para probar diferentes tipos de algoritmos de recomendaciones, También se muestra el rendimiento obtenido del simulador
	\item En l capítulo 4 se muestran las conclusione del proyecto y el posible trabajo futuro de cada a mejorar el simulador 
\end{itemize}